\chapter{การ{\wbr}อุปนัย{\wbr}เชิง{\wbr}คณิตศาสตร์}

ใน{\wbr}การ{\wbr}พัฒนา{\wbr}โปรแกรม{\wbr}ใด ๆ สิ่ง{\wbr}ที่{\wbr}เรา{\wbr}ต้อง{\wbr}สนใจ{\wbr}ก่อน{\wbr}ที่{\wbr}จะ{\wbr}พิจารณา{\wbr}ถึง{\wbr}ประสิทธิภาพ{\wbr}ของ{\wbr}โปรแกรม{\wbr}
ก็{\wbr}คือ{\wbr}ความ{\wbr}ถูกต้อง{\wbr}ของ{\wbr}โปรแกรม{\wbr}นั้น ๆ อย่างไรก็ตาม{\wbr}
การ{\wbr}พิสูจน์{\wbr}ความ{\wbr}ถูกต้อง{\wbr}ของ{\wbr}โปรแกรมย่อย{\wbr}แบบ{\wbr}เรียก{\wbr}ตัวเอง{\wbr}นั้น{\wbr}มี{\wbr}ความ{\wbr}ซับซ้อน{\wbr}เป็นพิเศษ{\wbr}
ขั้นตอน{\wbr}การ{\wbr}ทำงาน{\wbr}ทั้งหมด{\wbr}มัก{\wbr}ไม่{\wbr}ได้{\wbr}ถูก{\wbr}ระบุ{\wbr}ออก{\wbr}มา{\wbr}อย่าง{\wbr}ชัดเจน{\wbr}ภายใน{\wbr}โปรแกรมย่อย{\wbr}นั้น{\wbr}
แต่{\wbr}จะ{\wbr}เป็น{\wbr}การ{\wbr}โยน{\wbr}ภาระ{\wbr}การ{\wbr}ทำงาน{\wbr}ให้{\wbr}กับ{\wbr}โปรแกรมย่อย{\wbr}นั้น{\wbr}เอง{\wbr}

ใน{\wbr}ส่วน{\wbr}นี้{\wbr}เรา{\wbr}จะ{\wbr}ศึกษา{\wbr}เกี่ยวกับ{\wbr}เทคนิค{\wbr}การ{\wbr}พิสูจน์{\wbr}ที่{\wbr}เรียก{\wbr}ว่า {\em การ{\wbr}อุปนัย{\wbr}ทาง{\wbr}คณิตศาสตร์}
(mathematical induction)
ซึ่ง{\wbr}เป็น{\wbr}เครื่องมือ{\wbr}สำคัญ{\wbr}ใน{\wbr}การ{\wbr}พิสูจน์{\wbr}ความ{\wbr}ถูกต้อง{\wbr}ของ{\wbr}โปรแกรมย่อย{\wbr}แบบ{\wbr}เรียก{\wbr}ตัวเอง{\wbr}

เรา{\wbr}จะ{\wbr}เริ่ม{\wbr}จาก{\wbr}ตัวอย่าง พิจารณา{\wbr}ผลรวม $1+2+3+\cdots+n$ ถ้า{\wbr}ยัง{\wbr}พอ{\wbr}จำ{\wbr}ได้{\wbr}
ผลรวม{\wbr}ดังกล่าว{\wbr}มี{\wbr}ค่า{\wbr}เท่า{\wbr}กับ $\frac{n(n+1)}{2}$
อย่างไรก็ตาม{\wbr}เรา{\wbr}จะ{\wbr}พิสูจน์{\wbr}ได้{\wbr}อย่างไร{\wbr}ว่า{\wbr}ผลรวม{\wbr}ดังกล่าว{\wbr}มี{\wbr}ค่า{\wbr}เท่า{\wbr}กับ{\wbr}นิพจน์{\wbr}ที่{\wbr}อ้าง{\wbr}มา{\wbr}จริง{\wbr}

เรา{\wbr}อาจ{\wbr}ทดลอง{\wbr}ได้{\wbr}โดย{\wbr}การ{\wbr}แทน{\wbr}ค่า $ n $ ด้วย{\wbr}ค่า{\wbr}ต่าง ๆ เช่น{\wbr}

\begin{itemize}
\item เมื่อ $ n=1 $ เรา{\wbr}จะ{\wbr}ได้{\wbr}ว่า{\wbr}ผลรวม{\wbr}ข้างต้น{\wbr}มี{\wbr}ค่า{\wbr}เท่า{\wbr}กับ $ 1 $ ใน{\wbr}ขณะที่{\wbr}นิพจน์ $
  \frac{n(n+1)}{2}=\frac{1\cdot 2}{2}=1 $ เช่นกัน{\wbr}
\item เมื่อ $ n=2 $ เรา{\wbr}จะ{\wbr}ได้{\wbr}ว่า ผลรวม{\wbr}มี{\wbr}ค่า $ 1+2=3 $ ใน{\wbr}ขณะที่ $
  \frac{n(n+1)}{2}=\frac{2\cdot 3}{2}=3 $ เช่นกัน{\wbr}
\end{itemize}

เรา{\wbr}สามารถ{\wbr}ไล่{\wbr}แทน{\wbr}ค่า{\wbr}ไป{\wbr}ได้{\wbr}เรื่อย ๆ
... อย่างไรก็ตาม{\wbr}วิธี{\wbr}ดังกล่าว{\wbr}ไม่{\wbr}สามารถ{\wbr}พิสูจน์{\wbr}ได้{\wbr}ว่า{\wbr}ผลรวม{\wbr}มี{\wbr}ค่า{\wbr}เท่า{\wbr}กับ{\wbr}นิพจน์{\wbr}ที่{\wbr}อ้าง{\wbr}มา{\wbr}ได้{\wbr}
เนื่องจาก{\wbr}อาจ{\wbr}มี{\wbr}ค่า{\wbr}บาง{\wbr}ค่า{\wbr}ที่{\wbr}เรา{\wbr}ไม่{\wbr}ได้{\wbr}แทน{\wbr}ลง{\wbr}ไป{\wbr}ที่{\wbr}ผลรวม{\wbr}มี{\wbr}ค่า{\wbr}ไม่{\wbr}เท่า{\wbr}กับ{\wbr}นิพจน์{\wbr}ดังกล่าว{\wbr}

ใน{\wbr}การ{\wbr}พิสูจน์{\wbr}โดย{\wbr}ทั่วไป หลาย{\wbr}ครั้ง{\wbr}เรา{\wbr}จะ{\wbr}เริ่ม{\wbr}จาก{\wbr}สิ่ง{\wbr}ที่{\wbr}เรา{\wbr}ทราบ{\wbr}
จากนั้น{\wbr}จะ{\wbr}ใช้{\wbr}เหตุผล{\wbr}เพื่อ{\wbr}เชื่อมโยง{\wbr}ให้{\wbr}ได้{\wbr}ผลลัพธ์{\wbr}ตาม{\wbr}ที่{\wbr}เรา{\wbr}ต้องการ{\wbr}
และ{\wbr}ใน{\wbr}บาง{\wbr}ครั้ง{\wbr}เรา{\wbr}ก็{\wbr}จะ{\wbr}เริ่ม{\wbr}จาก{\wbr}ผล{\wbr}ที่{\wbr}เรา{\wbr}ต้องการ{\wbr}
แล้ว{\wbr}จึง{\wbr}พยายาม{\wbr}โยง{\wbr}สิ่ง{\wbr}ที่{\wbr}เรา{\wbr}ทราบ{\wbr}มา{\wbr}หา{\wbr}ผล{\wbr}ดังกล่าว{\wbr}โดย{\wbr}ใช้{\wbr}ลำดับ{\wbr}การ{\wbr}ให้{\wbr}เหตุผล{\wbr}ที่{\wbr}ถูกต้อง{\wbr}

การ{\wbr}อุปนัย{\wbr}ทาง{\wbr}คณิตศาสตร์ นั้น{\wbr}มี{\wbr}ลักษณะ{\wbr}ที่{\wbr}ดู{\wbr}ผิวเผิน{\wbr}แล้ว{\wbr}มี{\wbr}ลักษณะ{\wbr}แตกต่าง{\wbr}ออก{\wbr}ไป{\wbr}
เรา{\wbr}จะ{\wbr}เริ่ม{\wbr}จาก{\wbr}ตัวอย่าง{\wbr}การ{\wbr}พิสูจน์{\wbr}ว่า{\wbr}นิพจน์{\wbr}ดังกล่าว{\wbr}ข้างต้น{\wbr}มี{\wbr}ค่า{\wbr}เท่า{\wbr}กับ{\wbr}ผลรวม{\wbr}ตาม{\wbr}ที่{\wbr}เรา{\wbr}อ้าง{\wbr}ไว้{\wbr}

{\bf ขั้น{\wbr}ที่ 1.} เรา{\wbr}จะ{\wbr}เริ่ม{\wbr}โดย{\wbr}การ{\wbr}ตรวจสอบ{\wbr}ว่า{\wbr}นิพจน์{\wbr}ดังกล่าว{\wbr}มี{\wbr}ค่า{\wbr}เท่า{\wbr}กับ{\wbr}ผลรวม{\wbr}เมื่อ $ n=1 $ ซึ่ง{\wbr}ขั้นตอน{\wbr}นี้{\wbr}เรา{\wbr}ได้{\wbr}ทำ{\wbr}ไป{\wbr}แล้ว{\wbr}ตอน{\wbr}ต้น{\wbr}

{\bf ขั้น{\wbr}ที่ 2a.} จากนั้น{\wbr}เรา{\wbr}สมมติ{\wbr}ว่า{\wbr}นิพจน์{\wbr}ดังกล่าว{\wbr}มี{\wbr}ค่า{\wbr}เท่า{\wbr}กับ{\wbr}ผลรวม{\wbr}เมื่อ $ n=n' $ สำหรับ $ n' $ ใด ๆ ที่{\wbr}มี{\wbr}ค่า{\wbr}มาก{\wbr}กว่า{\wbr}หรือ{\wbr}เท่า{\wbr}กับ $ 1 $ นั่น{\wbr}คือ{\wbr}
$$1+2+\cdots+n'=\frac{n'(n'+1)}{2}$$ 	

{\bf ขั้น{\wbr}ที่ 2b.} จาก{\wbr}ข้อสมมติ{\wbr}ดังกล่าว เรา{\wbr}จะ{\wbr}แสดง{\wbr}ว่า{\wbr}นิพจน์{\wbr}ดังกล่าว{\wbr}มี{\wbr}ค่า{\wbr}เท่า{\wbr}กับ{\wbr}ผลรวม{\wbr}เมื่อ $ n=n'+1 $ ด้วย{\wbr}
นั่น{\wbr}คือ{\wbr}เรา{\wbr}จะ{\wbr}พิสูจน์{\wbr}ว่า: $$ 1+2+\cdots+n'+(n'+1)=\frac{(n'+1)(n'+2)}{2} $$

จาก{\wbr}ข้อสมมติ{\wbr}ที่ $ n=n' $ เรา{\wbr}สามารถ{\wbr}พิสูจน์{\wbr}สมการ{\wbr}เป้าหมาย{\wbr}ได้{\wbr}ไม่{\wbr}ยาก{\wbr}นัก ดังนี้{\wbr}
\begin{eqnarray*}
1+2+\cdots+n'+(n'+1) &=& (1+2+\cdots+n')+(n'+1)\\
&=& \frac{n'(n'+1)}{2}+(n'+1) \ \ \ \mbox{(โดย{\wbr}การ{\wbr}แทน{\wbr}ค่า{\wbr}จาก{\wbr}ข้อสมมติ)}\\
&=& \frac{n'(n'+1)+2(n'+1)}{2} \ \ \ \mbox{(จัด{\wbr}พจน์)}\\
&=& \frac{(n'+1)(n'+2)}{2} \ \ \ \mbox{(ดึง{\wbr}พจน์{\wbr}ย่อย{\wbr}ร่วม)}
\end{eqnarray*}

ก่อน{\wbr}จะ{\wbr}พิจารณา{\wbr}ต่อไป{\wbr}เรา{\wbr}จะ{\wbr}ทบทวน (อย่าง{\wbr}ไม่{\wbr}เป็น{\wbr}รูปแบบ) ว่า{\wbr}ใน{\wbr}ขั้นตอน{\wbr}ทั้ง 2 ขั้น เรา{\wbr}ได้{\wbr}แสดง{\wbr}อะไร{\wbr}ไป{\wbr}บ้าง{\wbr}

\begin{itemize}
\item เรา{\wbr}แสดง{\wbr}ว่า{\wbr}นิพจน์{\wbr}ดังกล่าว{\wbr}เท่า{\wbr}กับ{\wbr}ผลรวม{\wbr}จริง{\wbr}เมื่อ $ n=1 $
\item เรา{\wbr}สมมติ{\wbr}ให้{\wbr}นิพจน์{\wbr}ดังกล่าว{\wbr}เท่า{\wbr}กับ{\wbr}ผลรวม{\wbr}เมื่อ $ n=n' $ สำหรับ $ n'\geq 1 $ ใด ๆ จากนั้น{\wbr}แสดง{\wbr}ว่า{\wbr}นิพจน์{\wbr}ดังกล่าว{\wbr}เท่า{\wbr}กับ{\wbr}ผลรวม{\wbr}เมื่อ $ n=n'+1 $ ขั้นตอน{\wbr}นี้{\wbr}กล่าว{\wbr}ใน{\wbr}อีก{\wbr}ทาง{\wbr}หนึ่ง{\wbr}ก็{\wbr}คือ{\wbr}การ{\wbr}พิสูจน์{\wbr}ว่า:

ถ้า นิพจน์{\wbr}ดังกล่าว{\wbr}เท่า{\wbr}กับ{\wbr}ผลรวม{\wbr}เมื่อ $ n=n' $ สำหรับ $ n'\geq 1 $ ใด ๆ แล้ว นิพจน์{\wbr}ดังกล่าว{\wbr}เท่า{\wbr}กับ{\wbr}ผลรวม{\wbr}เมื่อ $ n=n'+1 $ ด้วย{\wbr}
\end{itemize}

ความจริง{\wbr}ทั้ง{\wbr}สอง{\wbr}ข้อ{\wbr}เมื่อ{\wbr}นำมา{\wbr}รวม{\wbr}กัน{\wbr}กลับ{\wbr}มี{\wbr}พลัง{\wbr}ใน{\wbr}การ{\wbr}พิสูจน์{\wbr}มากมาย กล่าวคือ{\wbr}
เมื่อ{\wbr}เรา{\wbr}ทราบ{\wbr}ว่า{\wbr}นิพจน์{\wbr}เท่า{\wbr}กับ{\wbr}ผลรวม{\wbr}เมื่อ $ n=1 $ (จาก{\wbr}ข้อ 1.)
เรา{\wbr}สามารถ{\wbr}นำ{\wbr}ความจริง{\wbr}ที่{\wbr}พิสูจน์{\wbr}ใน{\wbr}ข้อ 2 มา{\wbr}ใช้ได้ โดย{\wbr}พิจารณา{\wbr}กรณี{\wbr}ที่ $ n'=1 $
ดังนั้น{\wbr}ความจริง{\wbr}ข้อ 2 ทำ{\wbr}ให้{\wbr}เรา{\wbr}สรุป{\wbr}ได้{\wbr}ว่า นิพจน์{\wbr}เท่า{\wbr}กับ{\wbr}ผลรวม{\wbr}เมื่อ $ n=n'+1=2 $

ถ้า{\wbr}เรา{\wbr}เอา{\wbr}ความจริง{\wbr}ข้อ 2 มา{\wbr}ใช้{\wbr}อีก{\wbr}รอบ โดย{\wbr}ครั้งนี้{\wbr}ให้ $ n'=2 $
(สังเกต{\wbr}ว่า{\wbr}เรา{\wbr}สามารถ{\wbr}ความจริง{\wbr}ข้อ 2 มา{\wbr}ใช้ได้{\wbr}เนื่องจาก{\wbr}เรา{\wbr}ทราบ{\wbr}ว่า{\wbr}ผลรวม{\wbr}เท่า{\wbr}กับ{\wbr}นิพจน์{\wbr}เมื่อ $
n=2 $ แล้ว) เรา{\wbr}จะ{\wbr}สามารถ{\wbr}สรุป{\wbr}ได้{\wbr}ว่า นิพจน์{\wbr}เท่า{\wbr}กับ{\wbr}ผลรวม{\wbr}เมื่อ $ n=n'+1=3 $ ด้วย{\wbr}

เรา{\wbr}สามารถ{\wbr}นำ{\wbr}ความจริง{\wbr}ข้อ 2 มา{\wbr}ใช้{\wbr}ไป{\wbr}เรื่อย ๆ ทำ{\wbr}ให้{\wbr}เรา{\wbr}สามารถ{\wbr}อ้าง{\wbr}ไป{\wbr}ได้{\wbr}เรื่อย ๆ
ว่า{\wbr}นิพจน์{\wbr}นั้น{\wbr}เท่า{\wbr}กับ{\wbr}ผลรวม{\wbr}เมื่อ $ n=4,5,6,7,\ldots $
นั่น{\wbr}คือ{\wbr}เรา{\wbr}สามารถ{\wbr}สรุป{\wbr}ได้{\wbr}ว่า{\wbr}นิพจน์{\wbr}ดังกล่าว{\wbr}มี{\wbr}ค่า{\wbr}เท่า{\wbr}กับ{\wbr}ผลรวม{\wbr}สำหรับ{\wbr}ทุก ๆ จำนวนเต็ม $ n $ ที่{\wbr}
$ n\geq 1 $.

การ{\wbr}อ้าง{\wbr}ความจริง{\wbr}สอง{\wbr}ข้อ{\wbr}เพื่อ{\wbr}สรุป{\wbr}ใน{\wbr}ย่อหน้า{\wbr}ก่อน{\wbr}นั้น{\wbr}ค่อนข้าง{\wbr}จะ{\wbr}เป็น{\wbr}การ{\wbr}สรุป{\wbr}ที่{\wbr}ไม่{\wbr}เป็น{\wbr}รูปแบบ{\wbr}ทางการ{\wbr}นัก ก่อน{\wbr}ที่{\wbr}จะ{\wbr}เขียน{\wbr}อย่าง{\wbr}เป็นทางการ เรา{\wbr}จะ{\wbr}แนะนำ{\wbr}หลักการ{\wbr}อุปนัย{\wbr}เชิง{\wbr}คณิตศาสตร์{\wbr}

    หลักการ{\wbr}อุปนัย{\wbr}ทาง{\wbr}คณิตศาสตร์{\wbr}
    สำหรับ{\wbr}เพรดิเคต $ P(n) $ ที่{\wbr}ขึ้น{\wbr}กับ{\wbr}จำนวนนับ $ n $ ถ้า{\wbr}เรา{\wbr}สามารถ{\wbr}พิสูจน์{\wbr}ได้{\wbr}ว่า{\wbr}
    1. $ P(1) $ จริง และ{\wbr}
    2. สำหรับ{\wbr}จำนวนนับ $ i\geq 1 $ ใด ๆ $ P(i)\Rightarrow P(i+1) $
    แล้ว $ P(n) $ เป็นจริง{\wbr}สำหรับ{\wbr}ทุก ๆ จำนวนนับ $ n $

เรา{\wbr}เรียก{\wbr}ขั้น{\wbr}ที่ 1 ว่า{\em ขั้น{\wbr}ฐาน} (basis) และ{\wbr}เรียก{\wbr}ส่วน{\wbr}ที่{\wbr}สอง{\wbr}ว่า{\em ขั้น{\wbr}อุปนัย}
(inductive step) ใน{\wbr}การ{\wbr}พิสูจน์{\wbr}ขั้น{\wbr}อุปนัย เรา{\wbr}เริ่ม{\wbr}โดย{\wbr}การ{\wbr}สมมติ{\wbr}ว่า $ P(i) $
เป็นจริง{\wbr}สำหรับ{\wbr}จำนวนเต็ม $ i $ ใด ๆ ข้อสมมติ{\wbr}ดังกล่าว{\wbr}เรียก{\wbr}ว่า{\em สมมติฐาน{\wbr}การ{\wbr}อุปนัย}
(induction hypothesis)

สังเกต{\wbr}ว่า{\wbr}ใน{\wbr}ตัวอย่าง{\wbr}ที่{\wbr}ยัง{\wbr}ไม่{\wbr}สมบูรณ์{\wbr}ของ{\wbr}เรา{\wbr}นั้น เรา{\wbr}ได้{\wbr}พิสูจน์{\wbr}ขั้น{\wbr}ที่ 1 และ{\wbr}ขั้น{\wbr}ที่ 2 ไป{\wbr}แล้ว{\wbr}
โดย{\wbr}เพรดิเคต{\wbr}ที่{\wbr}เรา{\wbr}พิสูจน์ $ P(n) $ คือ{\wbr}

\begin{center}
``$ 1+2+\cdots+n = \frac{n(n+1)}{2} $''
\end{center}

ดังนั้น{\wbr}จาก{\wbr}หลักการ{\wbr}อุปนัย{\wbr}ทาง{\wbr}คณิตศาสตร์ เรา{\wbr}สามารถ{\wbr}สรุป{\wbr}ได้{\wbr}ว่า $ P(n) $ จริง{\wbr}สำหรับ{\wbr}ทุก ๆ
จำนวนนับ $ n $

ใน{\wbr}ส่วนต่อ{\wbr}ไป{\wbr}ของ{\wbr}บท{\wbr}นี้{\wbr}เรา{\wbr}จะ{\wbr}ได้{\wbr}ดู{\wbr}ตัวอย่าง{\wbr}การ{\wbr}พิสูจน์{\wbr}ด้วย{\wbr}การ{\wbr}อุปนัย{\wbr}ทาง{\wbr}คณิตศาสตร์{\wbr}อีก{\wbr}หลาย ๆ ตัวอย่าง{\wbr}

\subsection{ตัวอย่าง{\wbr}การ{\wbr}พิสูจน์{\wbr}พื้นฐาน}

\subsection{การ{\wbr}อุปนัย{\wbr}แบบ{\wbr}แข็งแกร่ง}

\subsection{การ{\wbr}พิสูจน์{\wbr}ขั้นตอนวิธี}

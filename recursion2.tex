\chapter{การ{\wbr}เรียก{\wbr}ตัวเอง: ตัวอย่าง}

\section{การ{\wbr}จัดเรียง{\wbr}ข้อมูล}

เรา{\wbr}มี{\wbr}จำนวนเต็ม $n$ จำนวน อยู่{\wbr}ใน{\wbr}อาร์เรย์ $x$ (นั่น{\wbr}คือ{\wbr}ข้อมูล{\wbr}คือ{\wbr}
$x[0],x[1],\ldots,x[n-1]$) เรา{\wbr}ต้องการ{\wbr}เรียง{\wbr}ข้อมูล{\wbr}ใน{\wbr}อาร์เรย์{\wbr}ดังกล่าว{\wbr}

\begin{quiz}{กรณี{\wbr}ง่าย}
มี{\wbr}กรณี{\wbr}ใด{\wbr}บ้าง{\wbr}ที่{\wbr}เรา{\wbr}สามารถ{\wbr}เรียง{\wbr}ข้อมูล{\wbr}ใน{\wbr}อาร์เรย์ $x$ ได้{\wbr}ง่าย{\wbr}มาก{\wbr}
\end{quiz}

กรณี{\wbr}ที่{\wbr}อาร์เรย์{\wbr}มี{\wbr}ข้อมูล{\wbr}เพียง 1 จำนวน การ{\wbr}จัดเรียง{\wbr}อาร์เรย์{\wbr}ดังกล่าว{\wbr}สามารถ{\wbr}ทำ{\wbr}ได้{\wbr}โดย{\wbr}ง่าย นั่น{\wbr}คือ{\wbr}
ไม่{\wbr}ต้อง{\wbr}ทำ{\wbr}อะไร{\wbr}เลย  ถ้า{\wbr}ไม่{\wbr}ใช่{\wbr}กรณี{\wbr}ดังกล่าว เรา{\wbr}จะ{\wbr}เรียง{\wbr}ข้อมูล{\wbr}ได้{\wbr}อย่างไร?

แทน{\wbr}ที่{\wbr}จะ{\wbr}เริ่ม{\wbr}แบบ{\wbr}ไม่{\wbr}มี{\wbr}อะไร{\wbr}เลย เรา{\wbr}จะ{\wbr}สมมติ{\wbr}ว่า{\wbr}ใน{\wbr}การ{\wbr}พัฒนา{\wbr}โปรแกรม{\wbr}ใน{\wbr}การ{\wbr}จัดการ{\wbr}เรียง{\wbr}ข้อมูล{\wbr}
$n$ ตัว เรา{\wbr}สามารถ{\wbr}เรียง{\wbr}ข้อมูล{\wbr}ใน{\wbr}อาร์เรย์{\wbr}ที่{\wbr}มี{\wbr}ข้อมูล{\wbr}จำนวน $m$ ตัว{\wbr}ได้{\wbr}

สังเกต{\wbr}ว่า{\wbr}เรา{\wbr}กำลัง{\wbr}แก้{\wbr}ปัญหา{\wbr}การ{\wbr}จัดเรียง{\wbr}ข้อมูล แต่{\wbr}เรา{\wbr}สมมติ{\wbr}ว่า{\wbr}เรา{\wbr}แก้{\wbr}ปัญหา{\wbr}นั้น{\wbr}ได้{\wbr}แล้ว!
ถ้า{\wbr}สมมติ{\wbr}กัน{\wbr}ได้{\wbr}ง่าย ๆ แบบ{\wbr}นี้{\wbr}โปรแกรม{\wbr}ที่{\wbr}ใช้{\wbr}เรียง{\wbr}ข้อมูล{\wbr}ของ{\wbr}เรา{\wbr}คง{\wbr}มี{\wbr}ลักษณะ{\wbr}ดังนี้{\wbr}

\begin{algt}
\noindent {\bf เรียง{\wbr}ข้อมูล{\wbr}ใน{\wbr}อาร์เรย์ $x$ ที่{\wbr}มี{\wbr}ข้อมูล $n$ ตัว}\\
\hspace*{0.2in} เรียง{\wbr}ข้อมูล{\wbr}ใน{\wbr}อาร์เรย์ $x$ ที่{\wbr}มี{\wbr}ข้อมูล $n$ ตัว\\
\hspace*{0.2in} คืน{\wbr}ผลลัพธ์{\wbr}
\end{algt}

\begin{quiz}{อัล{\wbr}กอ{\wbr}ริ{\wbr}ทึม{\wbr}สมมติ}
อัล{\wbr}กอ{\wbr}ริ{\wbr}ทึม{\wbr}ดังกล่าว{\wbr}ทำงาน{\wbr}ได้{\wbr}จริง{\wbr}หรือ{\wbr}ไม่? เพราะ{\wbr}เหตุใด?
\end{quiz}


\section{ตัวหารร่วมมาก}

ใน{\wbr}ส่วน{\wbr}นี้{\wbr}เรา{\wbr}จะ{\wbr}พัฒนา{\wbr}โปรแกรม{\wbr}เพื่อ{\wbr}คำนวณ{\wbr}ค่าตัว{\wbr}หารร่วมมาก ซึ่ง{\wbr}มี{\wbr}นิยาม{\wbr}ดังต่อไปนี้{\wbr}

สำหรับ{\wbr}จำนวนเต็ม{\wbr}สอง{\wbr}จำนวน $a$ และ $b$ เรา{\wbr}จะ{\wbr}กล่าว{\wbr}ว่า{\wbr}จำนวนเต็ม $c$ เป็น {\em
ตัวหาร{\wbr}ร่วม{\wbr}ของ $a$ และ $b$} ถ้า $c$ หาร $a$ ลงตัว และ $c$ หาร $b$ ลงตัว{\wbr}

สำหรับ{\wbr}จำนวนเต็ม{\wbr}สอง{\wbr}จำนวน $a$ และ $b$ {\em ตัวหารร่วมมาก} (เขียน{\wbr}ย่อ{\wbr}ว่า ห.{\wbr}ร.{\wbr}ม
ใน{\wbr}ภาษาอังกฤษ{\wbr}คือ Greatest common divisor หรือ{\wbr}ย่อ{\wbr}ว่า GCD)  คือ{\wbr}
ตัวหาร{\wbr}ร่วม{\wbr}ที่{\wbr}มี{\wbr}ค่า{\wbr}มาก{\wbr}ที่สุด{\wbr}

\begin{quiz}{ทบทวน{\wbr}ความ{\wbr}รู้}
ตัวหารร่วมมาก{\wbr}ของ 100 และ 60 คือ{\wbr}อะไร? \\
ตัวหารร่วมมาก{\wbr}ของ 100 และ 10 คือ{\wbr}อะไร?
\end{quiz}

เรา{\wbr}จะ{\wbr}หา{\wbr}ห.{\wbr}ร.{\wbr}ม ของ{\wbr}จำนวนเต็ม{\wbr}ไม่{\wbr}เป็น{\wbr}ลบ $a$ และ $b$ เพื่อ{\wbr}ความ{\wbr}ง่าย{\wbr}เรา{\wbr}จะ{\wbr}สมมติ{\wbr}ให้ $b$
มี{\wbr}ค่า{\wbr}น้อย{\wbr}กว่า{\wbr}หรือ{\wbr}เท่า{\wbr}กับ $a$

\begin{quiz}{กรณี{\wbr}ง่าย}
มี{\wbr}กรณี{\wbr}ใด{\wbr}บ้าง{\wbr}ที่{\wbr}เรา{\wbr}สามารถ{\wbr}คำนวณ{\wbr}ค่า{\wbr}ห.{\wbr}ร.{\wbr}ม ของ $a$ และ $b$ ได้{\wbr}ง่าย{\wbr}มาก{\wbr}
\end{quiz}




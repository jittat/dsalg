\chapter{อาร์เรย์ พอยน์เตอร์ และ{\wbr}การ{\wbr}วิเคราะห์{\wbr}ความ{\wbr}ซับซ้อน}

ใน{\wbr}บท{\wbr}นี้{\wbr}เรา{\wbr}จะ{\wbr}พิจารณา{\wbr}โครงสร้าง{\wbr}ข้อมูล{\wbr}พื้นฐาน{\wbr}สำหรับ{\wbr}จัด{\wbr}เก็บ{\wbr}และ{\wbr}ประมวลผล{\wbr}ข้อมูล{\wbr}จำนวน{\wbr}มาก{\wbr}ที่{\wbr}เรียก{\wbr}ว่า{\em
อาร์เรย์} (array) รวม{\wbr}ไป{\wbr}ถึง{\wbr}ข้อมูล{\wbr}ประเภท{\em พอยน์เตอร์} (pointer)
ซึ่ง{\wbr}เก็บ{\wbr}ตำแหน่ง{\wbr}ภายใน{\wbr}หน่วยความจำ{\wbr}
โดย{\wbr}เรา{\wbr}จะ{\wbr}เริ่ม{\wbr}พิจารณา{\wbr}แนว{\wbr}คิด{\wbr}ของ{\wbr}โครงสร้าง{\wbr}ข้อมูล{\wbr}และ{\wbr}ชนิด{\wbr}ข้อมูล{\wbr}ดังกล่าว{\wbr}โดย{\wbr}ไม่{\wbr}ขึ้น{\wbr}กับ{\wbr}ภาษา{\wbr}โปรแกรม{\wbr}ที่{\wbr}ใช้{\wbr}
จากนั้น{\wbr}เรา{\wbr}จะ{\wbr}ศึกษา{\wbr}วิธีการ{\wbr}เขียน{\wbr}ใน{\wbr}ภาษา C/C++
และ{\wbr}ศึกษา{\wbr}ความ{\wbr}สัมพันธ์{\wbr}ระหว่าง{\wbr}พอยน์เตอร์{\wbr}และ{\wbr}อาร์เรย์{\wbr}ซึ่ง{\wbr}เป็นคุณ{\wbr}ลักษณะ{\wbr}เฉพาะที่{\wbr}มี{\wbr}ใน{\wbr}ภาษา C/C++

ใน{\wbr}ขณะเดียวกัน เรา{\wbr}จะ{\wbr}เริ่ม{\wbr}ศึกษา{\wbr}การ{\wbr}วิเคราะห์{\wbr}อัล{\wbr}กอ{\wbr}ริ{\wbr}ทึม โดย{\wbr}จะ{\wbr}เริ่ม{\wbr}ที่{\em
  การ{\wbr}วิเคราะห์{\wbr}ความ{\wbr}ซับซ้อน} (complexity analysis) และ{\em
  การ{\wbr}วิเคราะห์{\wbr}เชิง{\wbr}เส้น{\wbr}กำกับ} (asymptotic analysis) โดย{\wbr}ใช้{\wbr}สัญกรณ์{\wbr}โอ{\wbr}ใหญ่ (Big-O
notation)

\section{อาร์เรย์}
อาร์เรย์เป็น{\wbr}โครงสร้าง{\wbr}ข้อมูล{\wbr}ที่{\wbr}เก็บ{\wbr}กลุ่ม{\wbr}ของ{\wbr}ข้อมูล{\wbr}เป็น{\wbr}รายการ{\wbr}
โดย{\wbr}ที่{\wbr}ข้อมูล{\wbr}แต่ละ{\wbr}ตัว{\wbr}จะ{\wbr}ถูก{\wbr}เก็บ{\wbr}ต่อเนื่อง{\wbr}กัน{\wbr}ใน{\wbr}หน่วยความจำ และ{\wbr}ถูก{\wbr}อ้าง{\wbr}ถึง{\wbr}โดย{\wbr}ใช้{\wbr}ดัชนี (index)
ตัวอย่าง{\wbr}ง่าย ๆ ของ{\wbr}อาร์เรย์{\wbr}คือ{\wbr}รายการ{\wbr}ข้อมูล{\wbr}ด้าน{\wbr}ล่าง{\wbr}นี้{\wbr}

\begin{center}
2, 3, 5, 7, 11, 13, 17, 19, 23
\end{center}

ถ้า{\wbr}เรา{\wbr}เรียก{\wbr}รายการ{\wbr}ดังกล่าว{\wbr}ว่า{\wbr}รายการ $A$ และ{\wbr}อ้าง{\wbr}ถึง{\wbr}ข้อมูล{\wbr}แต่ละ{\wbr}ตัว{\wbr}ด้วย{\wbr}ดัชนี{\wbr}ที่{\wbr}เริ่มต้น{\wbr}ด้วย 0
ข้อมูล{\wbr}แต่ละ{\wbr}ตัว{\wbr}ใน{\wbr}รายการ{\wbr}จะ{\wbr}ถูก{\wbr}อ้าง{\wbr}ถึง{\wbr}ได้{\wbr}ดัง{\wbr}ตาราง{\wbr}ใน{\wbr}รูป{\wbr}ที่~\ref{fig:array-array-access}

\begin{figure}
\begin{center}
\begin{tabular}{|c|c|c|c|c|c|c|c|c|}
\hline
$A[0]$ & $A[1]$ & $A[2]$ & $A[3]$ & $A[4]$ & $A[5]$ & $A[6]$ & $A[7]$ & $A[8]$ \\
\hline
2 & 3 & 5 & 7 & 11 & 13 & 17 & 19 & 23\\
\hline
\end{tabular}
\end{center}
\caption{การ{\wbr}อ้าง{\wbr}ถึง{\wbr}ข้อมูล{\wbr}แต่ละ{\wbr}ตัว{\wbr}ใน{\wbr}อาร์เรย์ $A$}
\label{fig:array-array-access}
\end{figure}


\begin{quiz}{การ{\wbr}คำนวณ{\wbr}ค่า}
จง{\wbr}หา{\wbr}ผลลัพธ์{\wbr}ของ{\wbr}นิพจน์{\wbr}เหล่านี้ (1) $A[4]$, (2) $A[7]$, (3) $A[A[0]]$, 
(4) $A[A[A[0]]]$, (5) $A[200]$
\end{quiz}
\begin{quizans}
(1) 11, (2) 19, (3) 5, (4) 13, (5) ไม่{\wbr}มี{\wbr}ค่า (ดู{\wbr}อธิบาย{\wbr}เพิ่มเติม)
\end{quizans}

การ{\wbr}หา{\wbr}คำตอบ{\wbr}ของ{\wbr}คำถาม{\wbr}ที่ (3) นั้น จำเป็น{\wbr}ต้อง{\wbr}เข้าใจ{\wbr}ขั้นตอน{\wbr}การ{\wbr}คำนวณ{\wbr}ค่า{\wbr}ของ{\wbr}นิพจน์{\wbr}
เรา{\wbr}ต้องการ{\wbr}หา{\wbr}ค่า $A[A[0]]$ ดังนั้น{\wbr}เรา{\wbr}ต้องหา{\wbr}ค่า $A[0]$ ก่อน{\wbr}
เมื่อ{\wbr}พิจารณา{\wbr}ใน{\wbr}อาร์เรย์{\wbr}เรา{\wbr}พบ{\wbr}ว่า $A[0]$ คือ $2$ ดังนั้น จากนั้น{\wbr}เรา{\wbr}จึง{\wbr}พิจารณา{\wbr}ข้อมูล{\wbr}
$A[2]$ ใน{\wbr}อา{\wbr}รเรย์ ซึ่ง{\wbr}จะ{\wbr}ได้{\wbr}ค่า $5$

ใน{\wbr}การ{\wbr}ทำงาน{\wbr}จริง อาร์เรย์จะ{\wbr}เก็บ{\wbr}ใน{\wbr}หน่วยความจำ{\wbr}ที่{\wbr}ต่อเนื่อง{\wbr}กัน{\wbr}
และ{\wbr}มักจะ{\wbr}มี{\wbr}ขอบเขต{\wbr}ที่{\wbr}จำกัด{\wbr}และ{\wbr}ต้อง{\wbr}ระบุ{\wbr}เมื่อ{\wbr}เริ่ม{\wbr}ใช้ เช่น อาร์เรย์จำนวน 100 ช่อง หรือ{\wbr}
100000 ช่อง{\wbr}เป็นต้น รูป{\wbr}ที่~\ref{fig:array-array-in-mem}
แสดง{\wbr}ตัวอย่าง{\wbr}ของ{\wbr}การ{\wbr}เก็บ{\wbr}ข้อมูล{\wbr}ของ{\wbr}อาร์เรย์{\wbr}ใน{\wbr}หน่วยความจำ{\wbr}

\begin{figure}
TODO: ใส่{\wbr}รูป{\wbr}
\caption{การ{\wbr}เก็บ{\wbr}ข้อมูล{\wbr}ของ{\wbr}อาร์เรย์{\wbr}ใน{\wbr}หน่วยความจำ}
\label{fig:array-array-in-mem}
\end{figure}

คำถาม{\wbr}ที่ (5) เป็น{\wbr}การ{\wbr}อ้าง{\wbr}ถึง{\wbr}ข้อมูล{\wbr}ที่อยู่{\wbr}นอก{\wbr}ขอบเขต{\wbr}ของ{\wbr}อาร์เรย์
ซึ่ง{\wbr}ผลลัพธ์{\wbr}ที่{\wbr}ได้{\wbr}จะ{\wbr}ขึ้น{\wbr}กับ{\wbr}ภาษา{\wbr}โปรแกรม{\wbr}ที่{\wbr}ใช้ สำหรับ{\wbr}ภาษา C หรือ C++
ผลลัพธ์{\wbr}ที่{\wbr}ได้{\wbr}จะ{\wbr}ขึ้น{\wbr}กับ{\wbr}ข้อมูล{\wbr}ใน{\wbr}หน่วยความจำ{\wbr}ใน{\wbr}ตำแหน่ง{\wbr}ที่ $A[200]$ ควร{\wbr}จะ{\wbr}อยู่{\wbr}
เรา{\wbr}จะ{\wbr}ได้{\wbr}ศึกษา{\wbr}รายละเอียด{\wbr}นี้{\wbr}ต่อไป อย่างไรก็ตาม ปกติ{\wbr}แล้ว ใน{\wbr}การ{\wbr}ใช้{\wbr}งาน{\wbr}อาร์เรย์
เรา{\wbr}จะ{\wbr}ไม่{\wbr}อ้าง{\wbr}ถึง{\wbr}ข้อมูล{\wbr}ที่อยู่{\wbr}นอก{\wbr}ขอบเขต{\wbr}ของ{\wbr}อาร์เรย์

การ{\wbr}ระบุ{\wbr}ดัชนี{\wbr}ของ{\wbr}ข้อมูล{\wbr}ใน{\wbr}อาร์เรย์{\wbr}ใน{\wbr}หนังสือ{\wbr}เล่ม{\wbr}นี้{\wbr}จะ{\wbr}อ้างอิง{\wbr}จาก{\wbr}ภาษา{\wbr}ตระกูล{\wbr}ภาษา C
นั่น{\wbr}คือ{\wbr}เริ่มต้น{\wbr}ที่ 0 \ \ สำหรับ{\wbr}บาง{\wbr}ภาษา เรา{\wbr}สามารถ{\wbr}ระบุ{\wbr}ค่า{\wbr}เริ่มต้น{\wbr}ของ{\wbr}ดัชนี{\wbr}ได้{\wbr}และ{\wbr}มัก{\wbr}เริ่ม{\wbr}ที่ 1
เช่น{\wbr}ภาษา{\wbr}ปา{\wbr}ส{\wbr}คา{\wbr}ล (Pascal) เป็นต้น{\wbr}
อย่างไรก็ตาม{\wbr}แนว{\wbr}คิด{\wbr}ใน{\wbr}การ{\wbr}พัฒนา{\wbr}โปรแกรม{\wbr}นั้น{\wbr}จะ{\wbr}ไม่{\wbr}ต่าง{\wbr}กัน{\wbr}

เมื่อ{\wbr}เรา{\wbr}สามารถ{\wbr}อ้าง{\wbr}ถึง{\wbr}ข้อมูล{\wbr}ได้{\wbr}ด้วย{\wbr}ดัชนี{\wbr}
เรา{\wbr}สามารถ{\wbr}ใช้{\wbr}ตัวแปร{\wbr}เพื่อ{\wbr}แทน{\wbr}ค่า{\wbr}ดัชนี{\wbr}ของ{\wbr}ข้อมูล{\wbr}ที่{\wbr}เรา{\wbr}ต้องการ{\wbr}ใช้{\wbr}งาน{\wbr}ได้{\wbr}
ความ{\wbr}สามารถ{\wbr}นี้{\wbr}ทำ{\wbr}ให้{\wbr}เรา{\wbr}สามารถ{\wbr}เขียน{\wbr}โปรแกรม{\wbr}ที่{\wbr}มี{\wbr}ลักษณะ{\wbr}ดัง{\wbr}ด้าน{\wbr}ล่าง{\wbr}ได้{\wbr}

\begin{algt}
\label{alg:array-sum1}
\noindent \hspace*{0.2in} ให้ $x\leftarrow 0$\\
\hspace*{0.2in} พิจารณา ตัวแปร $i\leftarrow 0,1,\ldots,8$\\
\hspace*{0.2in}\hspace*{0.2in} ให้ $x \leftarrow x + A[i]$
\end{algt}

\begin{quiz}{}
อัล{\wbr}กอ{\wbr}ริ{\wbr}ทึม{\wbr}ดังกล่าว{\wbr}คำนวณ{\wbr}ค่า{\wbr}บาง{\wbr}อย่าง{\wbr}ใน{\wbr}ตัวแปร $x$ ค่า{\wbr}นั้น{\wbr}คือ{\wbr}อะไร?
\end{quiz}
\begin{quizans}
ผลรวม{\wbr}ของ{\wbr}ข้อมูล{\wbr}ทั้งหมด{\wbr}ใน{\wbr}อาร์เรย์ $A$
\end{quizans}

สังเกต{\wbr}ว่า{\wbr}อัล{\wbr}กอ{\wbr}ริ{\wbr}ทึม~\ref{alg:array-sum1} เขียน{\wbr}ให้{\wbr}ทำงาน{\wbr}กับ{\wbr}อาร์เรย์ $A$
ที่{\wbr}มี{\wbr}ดัชนี{\wbr}มาก{\wbr}ที่สุด{\wbr}คือ 8 เท่านั้น ใน{\wbr}การ{\wbr}พัฒนา{\wbr}อัล{\wbr}กอ{\wbr}ริ{\wbr}ทึม{\wbr}ทั่วไป{\wbr}เรา{\wbr}มัก{\wbr}เขียน{\wbr}ให้{\wbr}ทำงาน{\wbr}ได้{\wbr}กับ{\wbr}ข้อมูล{\wbr}ทั่วไป{\wbr}
ซึ่ง{\wbr}ใน{\wbr}กรณี{\wbr}นี้ การ{\wbr}จะ{\wbr}ปรับ{\wbr}ให้{\wbr}ทำงาน{\wbr}ได้{\wbr}กับ{\wbr}อาร์เรย์{\wbr}ใด ๆ เรา{\wbr}จะ{\wbr}ต้อง{\wbr}ระบุ{\wbr}ขนาด{\wbr}ของ{\wbr}อาร์เรย์{\wbr}ด้วย{\wbr}
เรา{\wbr}สามารถ{\wbr}เขียน{\wbr}อัล{\wbr}กอ{\wbr}ริ{\wbr}ทึม{\wbr}ดังกล่าว{\wbr}โดย{\wbr}ระบุ{\wbr}พารามิเตอร์{\wbr}ให้{\wbr}ชัดเจน{\wbr}ขึ้น{\wbr}ได้{\wbr}ดัง{\wbr}ด้าน{\wbr}ล่าง{\wbr}

\begin{algt}
\label{alg:array-sum2}
\noindent {\bf คำนวณ{\wbr}ค่า{\wbr}บาง{\wbr}อย่าง{\wbr}ของ{\wbr}อาร์เรย์ $A$ ที่{\wbr}มี{\wbr}ข้อมูล{\wbr}จำนวน $n$ ตัว}\\
\hspace*{0.2in} ให้ $x\leftarrow 0$\\
\hspace*{0.2in} พิจารณา ตัวแปร $i\leftarrow 0,1,\ldots,n-1$\\
\hspace*{0.2in}\hspace*{0.2in} ให้ $x \leftarrow x + A[i]$\\
\hspace*{0.2in} คืน{\wbr}ค่า $x$ เป็น{\wbr}คำตอบ{\wbr}
\end{algt}

\subsection{เวลา{\wbr}ที่{\wbr}ใช้{\wbr}ใน{\wbr}การ{\wbr}ทำงาน}

ค่า{\wbr}พารามิเตอร์ $n$ ที่{\wbr}เรา{\wbr}ส่ง{\wbr}ให้{\wbr}กับ{\wbr}โปรแกรมย่อย ระบุ{\wbr}จำนวน{\wbr}รอบ{\wbr}ของ{\wbr}การ{\wbr}ทำงาน{\wbr}
ซึ่ง{\wbr}จะ{\wbr}เป็น{\wbr}ตัวกำหนด{\wbr}เวลา{\wbr}ที่{\wbr}โปรแกรมย่อย{\wbr}ใช้{\wbr}ใน{\wbr}การ{\wbr}ทำงาน{\wbr}ด้วย อย่างไรก็ตาม{\wbr}
เพียงแค่{\wbr}พิจารณา{\wbr}โปรแกรมย่อย{\wbr}ดังกล่าว เรา{\wbr}ไม่{\wbr}สามารถ{\wbr}ระบุ{\wbr}เวลาจริง ๆ
ที่{\wbr}โปรแกรมย่อย{\wbr}จะ{\wbr}ทำงาน{\wbr}ได้{\wbr}เนื่องจาก{\wbr}เรา{\wbr}ไม่{\wbr}ทราบ{\wbr}ปัจจัย{\wbr}หลาย ๆ อย่าง{\wbr}

\begin{quiz}{เวลา{\wbr}การ{\wbr}ทำงาน{\wbr}จริง{\wbr}บน{\wbr}คอมพิวเตอร์}
ปัจจัย{\wbr}อะไร{\wbr}บ้าง{\wbr}ที่{\wbr}กำหนด{\wbr}เวลา{\wbr}ทำงาน{\wbr}บน{\wbr}คอมพิวเตอร์{\wbr}จริง ๆ ของ{\wbr}โปรแกรมย่อย{\wbr}ข้างต้น{\wbr}
\end{quiz}
\begin{quizans}
เวลา{\wbr}ใน{\wbr}การ{\wbr}ทำงาน{\wbr}จริง ขึ้น{\wbr}กับ (1) โปรแกรม{\wbr}ภาษาคอมพิวเตอร์{\wbr}ที่{\wbr}เขียน{\wbr}จาก{\wbr}โปรแกรมย่อย (2)
คอม{\wbr}ไพ{\wbr}เลอร์{\wbr}ที่{\wbr}ใช้ (3) เครื่อง{\wbr}คอมพิวเตอร์{\wbr}ที่{\wbr}นำ{\wbr}โปรแกรม{\wbr}ไป{\wbr}ทำงาน{\wbr}
และ{\wbr}สถานะ{\wbr}ของ{\wbr}เครื่องใน{\wbr}ขณะที่{\wbr}โปรแกรม{\wbr}ทำงาน  
\end{quizans}

สังเกต{\wbr}ว่า{\wbr}การ{\wbr}พิจารณา{\wbr}แค่{\wbr}อัล{\wbr}กอ{\wbr}ริ{\wbr}ทึม{\wbr}เพียง{\wbr}อย่างเดียว{\wbr}
หรือ{\wbr}กระทั่ง{\wbr}จะ{\wbr}พิจารณา{\wbr}โปรแกรม{\wbr}ใน{\wbr}ภาษาเครื่อง{\wbr}ที่{\wbr}ถูก{\wbr}คอมไพล์{\wbr}แล้ว{\wbr}ร่วม{\wbr}ด้วย{\wbr}
ก็{\wbr}ไม่{\wbr}สามารถ{\wbr}ทำ{\wbr}ให้{\wbr}เรา{\wbr}ระบุ{\wbr}เวลา{\wbr}การ{\wbr}ทำงาน{\wbr}บน{\wbr}คอมพิวเตอร์{\wbr}จริง{\wbr}ได้{\wbr}อย่าง{\wbr}แม่นยำ{\wbr}
ยิ่ง{\wbr}ใน{\wbr}ปัจจุบัน{\wbr}ที่{\wbr}คอมพิวเตอร์{\wbr}สามารถ{\wbr}ทำงาน{\wbr}หลาย ๆ งาน{\wbr}ใน{\wbr}เวลา{\wbr}เดียวกัน{\wbr}
การ{\wbr}ทำนาย{\wbr}เวลา{\wbr}การ{\wbr}ทำงาน{\wbr}จริง{\wbr}ยิ่ง{\wbr}กระทำ{\wbr}ได้{\wbr}ยาก{\wbr}ขึ้น{\wbr}ด้วย{\wbr}

อย่างไรก็ตาม แม้{\wbr}การ{\wbr}ระบุ{\wbr}เวลา{\wbr}การ{\wbr}ทำงาน{\wbr}จริง ๆ ทำ{\wbr}ได้{\wbr}ยาก{\wbr}
การ{\wbr}ทำนาย{\wbr}เวลา{\wbr}การ{\wbr}ทำงาน{\wbr}ของ{\wbr}อัล{\wbr}กอ{\wbr}ริ{\wbr}ทึม{\wbr}ก่อน{\wbr}ที่{\wbr}จะ{\wbr}นำ{\wbr}ไป{\wbr}พัฒนา{\wbr}เป็น{\wbr}โปรแกรม{\wbr}ก็{\wbr}ยัง{\wbr}เป็น{\wbr}สิ่ง{\wbr}จำเป็น{\wbr}มาก{\wbr}
เนื่องจาก{\wbr}ใน{\wbr}หลาย ๆ เรา{\wbr}สามารถ{\wbr}เลือก{\wbr}ใช้{\wbr}อัล{\wbr}กอ{\wbr}ริ{\wbr}ทึม{\wbr}ได้{\wbr}หลากหลาย{\wbr}
และ{\wbr}อัล{\wbr}กอ{\wbr}ริ{\wbr}ทึม{\wbr}เหล่านั้น{\wbr}ก็{\wbr}มี{\wbr}ความ{\wbr}ซับซ้อน{\wbr}ใน{\wbr}การ{\wbr}นำ{\wbr}ไป{\wbr}พัฒนา{\wbr}เป็น{\wbr}โปรแกรม{\wbr}ที่{\wbr}แตกต่าง{\wbr}กัน{\wbr}
โปรแกรมเมอร์{\wbr}จึง{\wbr}ต้อง{\wbr}เลือก{\wbr}ใช้{\wbr}อัล{\wbr}กอ{\wbr}ริ{\wbr}ทึม{\wbr}ให้{\wbr}เหมาะสม นั่น{\wbr}คือ{\wbr}เป็น{\wbr}อัล{\wbr}กอ{\wbr}ริ{\wbr}ทึม{\wbr}ที่{\wbr}เมื่อ{\wbr}นำ{\wbr}ไป{\wbr}พัฒนา{\wbr}แล้ว{\wbr}
มี{\wbr}ประสิทธิภาพ{\wbr}พอ (ทำงาน{\wbr}ได้{\wbr}ทัน{\wbr}เวลา)
และ{\wbr}มี{\wbr}ความ{\wbr}ซับซ้อน{\wbr}ใน{\wbr}การ{\wbr}เขียน{\wbr}ใน{\wbr}ระดับ{\wbr}ที่{\wbr}โปรแกรมเมอร์{\wbr}สามารถ{\wbr}จัดการ{\wbr}ได้{\wbr}
การ{\wbr}เลือก{\wbr}นำ{\wbr}อัล{\wbr}กอ{\wbr}ริ{\wbr}ทึม{\wbr}ที่{\wbr}ทราบ{\wbr}ว่า{\wbr}มี{\wbr}ประสิทธิภาพ{\wbr}ดี{\wbr}ที่สุด{\wbr}ไป{\wbr}พัฒนา{\wbr}นั้น{\wbr}
อาจ{\wbr}ไม่{\wbr}ใช่{\wbr}ทางเลือก{\wbr}ที่{\wbr}ดี{\wbr}ที่สุด{\wbr}ก็{\wbr}เป็น{\wbr}ได้{\wbr}

ดังนั้น เรา{\wbr}จะ{\wbr}พยายาม{\wbr}วิเคราะห์{\wbr}เวลา{\wbr}การ{\wbr}ทำงาน{\wbr}ของ{\wbr}โปรแกรมย่อย{\wbr}
ที่อยู่{\wbr}ใน{\wbr}รูป{\wbr}ของ{\wbr}โปรแกรม{\wbr}ลำ{\wbr}ลอง{\wbr}ด้าน{\wbr}บน ให้{\wbr}ละเอียด{\wbr}เท่า{\wbr}ที่{\wbr}เรา{\wbr}พอจะ{\wbr}ทำ{\wbr}ได้{\wbr}
แน่นอน{\wbr}เรา{\wbr}จำเป็น{\wbr}ต้อง{\wbr}เพิ่ม{\wbr}ข้อสมมติ{\wbr}หลาย{\wbr}อย่าง{\wbr}เพื่อให้{\wbr}การ{\wbr}วิเคราะห์{\wbr}เป็น{\wbr}ไป{\wbr}ได้{\wbr}

ข้อสมมติ{\wbr}ข้อ{\wbr}แรก (ที่{\wbr}เรา{\wbr}จะ{\wbr}ใช้{\wbr}ตลอด{\wbr}ใน{\wbr}หนังสือ{\wbr}เล่ม{\wbr}นี้) คือ{\wbr}
เรา{\wbr}จะ{\wbr}สมมติ{\wbr}ว่า{\wbr}คอมพิวเตอร์{\wbr}นั้น{\wbr}ทำงาน{\wbr}ทีละ{\wbr}คำสั่ง นั่น{\wbr}คือ{\wbr}ไม่{\wbr}ใช่{\wbr}คอมพิวเตอร์{\wbr}แบบ{\wbr}ขนาน{\wbr}
หรือ{\wbr}เป็น{\wbr}คอมพิวเตอร์{\wbr}ที่{\wbr}มี{\wbr}หน่วย{\wbr}ประมวลผล{\wbr}หลาย{\wbr}ตัว{\wbr}ทำงาน{\wbr}พร้อมกัน\footnote{TODO:
  ระบุ{\wbr}ว่า{\wbr}ถึง{\wbr}จะ{\wbr}เป็น{\wbr}กรณี{\wbr}ดังกล่าว การ{\wbr}วิเคราะห์{\wbr}ก็{\wbr}ยัง{\wbr}เป็น{\wbr}ไป{\wbr}ได้}

ถ้า{\wbr}พิจารณา{\wbr}ต่อไป เรา{\wbr}จะ{\wbr}พบ{\wbr}ว่า{\wbr}โปรแกรมย่อย~\ref{alg:array-sum2}
ทำงาน{\wbr}โดย{\wbr}ใช้เวลา{\wbr}ใน{\wbr}การ{\wbr}ทำงาน{\wbr}ที่{\wbr}แปรผัน{\wbr}ตาม{\wbr}ค่า{\wbr}พารามิเตอร์ $n$
เพื่อ{\wbr}จะ{\wbr}ให้{\wbr}เรา{\wbr}สามารถ{\wbr}วิเคราะห์{\wbr}เวลา{\wbr}การ{\wbr}ทำงาน{\wbr}ออก{\wbr}มา{\wbr}ได้{\wbr}
เรา{\wbr}จะ{\wbr}สมมติ{\wbr}ว่า{\wbr}คอมพิวเตอร์{\wbr}เมื่อ{\wbr}ทำงาน{\wbr}ตาม{\wbr}โปรแกรม{\wbr}ดังกล่าว ใช้เวลา 1
หน่วย{\wbr}ใน{\wbr}การ{\wbr}ประมวลผล{\wbr}คำสั่ง{\wbr}แต่ละ{\wbr}บรรทัด{\wbr}
เรา{\wbr}จะ{\wbr}สามารถ{\wbr}คำนวณ{\wbr}เวลา{\wbr}ที่{\wbr}โปรแกรม{\wbr}ดังกล่าว{\wbr}ใช้{\wbr}โดย{\wbr}พิจารณา{\wbr}จำนวน{\wbr}ครั้ง{\wbr}ที่{\wbr}คำสั่ง{\wbr}ใน{\wbr}แต่ละ{\wbr}บรรทัด{\wbr}ทำงาน{\wbr}
ดัง{\wbr}ด้าน{\wbr}ล่าง{\wbr}

\begin{algt}
\noindent \hspace*{0.2in} ให้ $x\leftarrow 0$   \ \ \ \ $\rhd\rhd\rhd$ ทำงาน 1 ครั้ง\\
\hspace*{0.2in} พิจารณา ตัวแปร $i\leftarrow 0,1,\ldots,n-1$  \ \ \ \ $\rhd\rhd\rhd$ ทำงาน $n$ ครั้ง\\
\hspace*{0.2in}\hspace*{0.2in} ให้ $x \leftarrow x + A[i]$  \ \ \ \ $\rhd\rhd\rhd$ ทำงาน $n$ ครั้ง\\
\hspace*{0.2in} คืน{\wbr}ค่า $x$ เป็น{\wbr}คำตอบ  \ \ \ \ $\rhd\rhd\rhd$ ทำงาน 1 ครั้ง{\wbr}
\end{algt}

ดังนั้น{\wbr}เรา{\wbr}จะ{\wbr}ได้{\wbr}ว่า{\wbr}เวลา{\wbr}รวม{\wbr}คือ $2n + 2$ หน่วย คำถาม{\wbr}ที่{\wbr}ตาม{\wbr}มา{\wbr}ก็{\wbr}คือ{\wbr}
ผลลัพธ์{\wbr}จาก{\wbr}การ{\wbr}วิเคราะห์{\wbr}ดังกล่าว{\wbr}มี{\wbr}ความ{\wbr}แม่นยำ และ{\wbr}สามารถ{\wbr}นำ{\wbr}ไป{\wbr}ใช้{\wbr}พิจารณา{\wbr}ต่อไป{\wbr}ได้{\wbr}เพียงใด{\wbr}

TODO: อธิบาย{\wbr}และ{\wbr}ยก{\wbr}ตัวอย่าง{\wbr}ถึง{\wbr}ความจำ{\wbr}เป็น{\wbr}ของ asymptotic analysis

\subsection{การ{\wbr}ประมวลผล{\wbr}รายการ{\wbr}ด้วย{\wbr}อาร์เรย์}

ใน{\wbr}ส่วน{\wbr}นี้{\wbr}เรา{\wbr}จะ{\wbr}พัฒนา{\wbr}โปรแกรม{\wbr}ลำ{\wbr}ลอง{\wbr}เพื่อ{\wbr}ประมวลผล{\wbr}ข้อมูล{\wbr}ใน{\wbr}รายการ{\wbr}ที่{\wbr}เก็บ{\wbr}ใน{\wbr}อาร์เรย์
พร้อมกับ{\wbr}วิเคราะห์{\wbr}เวลา{\wbr}การ{\wbr}ทำงาน{\wbr}

\begin{quiz}{}
สมมติ{\wbr}ว่า{\wbr}เรา{\wbr}มี{\wbr}รายการ{\wbr}ของ{\wbr}ข้อมูล ลอง{\wbr}นึก{\wbr}ตัวอย่าง{\wbr}การ{\wbr}ประมวลผล{\wbr}ที่{\wbr}เรา{\wbr}สามารถ{\wbr}กระทำ{\wbr}กับ{\wbr}ข้อมูล{\wbr}ใน{\wbr}รายการ{\wbr}นี้{\wbr}
\end{quiz}

ก่อน{\wbr}ที่{\wbr}เรา{\wbr}จะ{\wbr}ประมวลผล{\wbr}ได้ เรา{\wbr}ต้อง{\wbr}พิจารณา{\wbr}วิธีการ{\wbr}จัด{\wbr}เก็บ{\wbr}ข้อมูล{\wbr}แบบ{\wbr}รายการ{\wbr}ลง{\wbr}ใน{\wbr}อาร์เรย์{\wbr}ก่อน{\wbr}
สังเกต{\wbr}ว่า{\wbr}โครงสร้าง{\wbr}ข้อมูล{\wbr}แบบ{\wbr}อาร์เรย์{\wbr}มี{\wbr}ลักษณะ{\wbr}เป็น{\wbr}รายการ{\wbr}อยู่{\wbr}แล้ว{\wbr}
อย่างไรก็ตาม{\wbr}ใน{\wbr}การ{\wbr}จัดการ{\wbr}กับ{\wbr}รายการ{\wbr}ที่{\wbr}มี{\wbr}จำนวน{\wbr}ข้อมูล{\wbr}เปลี่ยนแปลง{\wbr}ได้{\wbr}
การ{\wbr}ใช้{\wbr}อาร์เรย์{\wbr}เพียง{\wbr}อย่างเดียว{\wbr}นั้น{\wbr}ไม่{\wbr}เพียงพอ{\wbr}

\begin{quiz}{}
อะไร{\wbr}คือ{\wbr}สิ่ง{\wbr}ที่{\wbr}ขาด{\wbr}หาย{\wbr}ไป ถ้า{\wbr}เรา{\wbr}ใช้{\wbr}แค่{\wbr}อาร์เรย์{\wbr}ใน{\wbr}การ{\wbr}จัด{\wbr}เก็บ{\wbr}รายการ{\wbr}ที่{\wbr}จำนวน{\wbr}ข้อมูล{\wbr}ใน{\wbr}รายการ{\wbr}เปลี่ยนแปลง{\wbr}ได้{\wbr}
\end{quiz}

ดังนั้น เรา{\wbr}จะ{\wbr}ใช้{\wbr}ตัวแปร{\wbr}อีก{\wbr}หนึ่ง{\wbr}ตัว{\wbr}ใน{\wbr}การ{\wbr}เก็บ{\wbr}จำนวน{\wbr}ข้อมูล{\wbr}ที่{\wbr}มี{\wbr}ใน{\wbr}อาร์เรย์
โปรแกรม{\wbr}ลำ{\wbr}ลอง{\wbr}ที่{\wbr}เรา{\wbr}จะ{\wbr}พัฒนา{\wbr}จะ{\wbr}เปลี่ยน{\wbr}ค่า{\wbr}ของ{\wbr}ตัวแปร{\wbr}นี้{\wbr}โดย{\wbr}ตรง{\wbr}เพื่อ{\wbr}ปรับ{\wbr}ให้{\wbr}มี{\wbr}ค่า{\wbr}ที่{\wbr}ถูกต้อง{\wbr}ภายหลัง{\wbr}การ{\wbr}ประมวลผล{\wbr}
ใน{\wbr}การ{\wbr}พัฒนา{\wbr}โปรแกรม{\wbr}ลำ{\wbr}ลอง{\wbr}ให้{\wbr}เป็น{\wbr}โปรแกรม{\wbr}ภาษา C/C++
การ{\wbr}ทำงาน{\wbr}ดังกล่าว{\wbr}จะ{\wbr}ต้อง{\wbr}ใช้{\wbr}การ{\wbr}ส่ง{\wbr}รับ{\wbr}พารามิเตอร์{\wbr}เป็น{\wbr}พอยน์เตอร์{\wbr}หรือ{\wbr}ส่ง{\wbr}แบบ pass by
reference ซึ่ง{\wbr}เรา{\wbr}จะ{\wbr}ได้{\wbr}พิจารณา{\wbr}ใน{\wbr}ส่วน~\ref{sect:array-pointer-c}
นอกจากนี้{\wbr}ใน{\wbr}บท{\wbr}ที่~\ref{chapter:class} เรา{\wbr}จะ{\wbr}ได้{\wbr}ศึกษา{\wbr}วิธีการ{\wbr}ที่{\wbr}จะ ``ประกอบ{\wbr}รวม''
อาร์เรย์และ{\wbr}ตัวแปร{\wbr}ที่{\wbr}เก็บ{\wbr}จำนวน{\wbr}ข้อมูล{\wbr}ที่อยู่{\wbr}ใน{\wbr}อาร์เรย์{\wbr}เข้า{\wbr}ด้วย{\wbr}กัน{\wbr}
เพื่อ{\wbr}สร้าง{\wbr}เป็น{\wbr}ชนิด{\wbr}ข้อมูล{\wbr}ใหม่{\wbr}ที่{\wbr}นำ{\wbr}ไป{\wbr}ใช้{\wbr}งาน{\wbr}ได้{\wbr}สะดวก{\wbr}ต่อไป{\wbr}

เรา{\wbr}จะ{\wbr}พิจารณา{\wbr}การ{\wbr}ประมวลผล{\wbr}กับ{\wbr}อาร์เรย์{\wbr}ใน{\wbr}รูปแบบ{\wbr}ต่าง ๆ ดังนี้ (1) การ{\wbr}ค้น{\wbr}ข้อมูล{\wbr}ใน{\wbr}รายการ, (2) การ{\wbr}เพิ่ม{\wbr}ข้อมูล{\wbr}ลง{\wbr}ไป{\wbr}ตอน{\wbr}ท้าย{\wbr}ของ{\wbr}รายการ, (3) การ{\wbr}ลบ{\wbr}ข้อมูล{\wbr}ใน{\wbr}รายการ, และ (4) การ{\wbr}แทรก{\wbr}ข้อมูล{\wbr}ใน{\wbr}รายการ{\wbr}

\subsubsection{การ{\wbr}ค้น{\wbr}ข้อมูล} 
สำหรับ{\wbr}การ{\wbr}ค้น{\wbr}ข้อมูล{\wbr}ใน{\wbr}รายการ{\wbr}
เป้าหมาย{\wbr}ของ{\wbr}การ{\wbr}ทำงาน{\wbr}คือ{\wbr}ทราบ{\wbr}ว่า{\wbr}มี{\wbr}ข้อมูล{\wbr}ที่{\wbr}เรา{\wbr}ต้องการ{\wbr}หา{\wbr}หรือ{\wbr}ไม่ และ{\wbr}ถ้า{\wbr}มี{\wbr}อยู่{\wbr}ที่{\wbr}ตำแหน่ง{\wbr}ใด{\wbr}
ใน{\wbr}กรณี{\wbr}นี้{\wbr}เรา{\wbr}จะ{\wbr}ต้อง{\wbr}พิจารณา{\wbr}ข้อมูล{\wbr}ทุก{\wbr}ตัว{\wbr}ใน{\wbr}รายการ{\wbr}
โปรแกรม{\wbr}ลำ{\wbr}ลอง{\wbr}มี{\wbr}ลักษณะ{\wbr}ไม่{\wbr}ต่าง{\wbr}จาก{\wbr}ที่{\wbr}เรา{\wbr}เคย{\wbr}เขียน{\wbr}เท่าใด{\wbr}นัก{\wbr}

\begin{algt}
\noindent {\bf ค้นหา{\wbr}ข้อมูล $x$ ใน{\wbr}อาร์เรย์ $A$ ที่{\wbr}มี{\wbr}ข้อมูล{\wbr}จำนวน $n$ ตัว}\\
\hspace*{0.2in} พิจารณา ตัวแปร $i\leftarrow 0,1,\ldots, n-1$\\
\hspace*{0.2in}\hspace*{0.2in} ถ้า $A[i] = x$\\
\hspace*{0.2in}\hspace*{0.2in}\hspace*{0.2in} คืน{\wbr}ค่า $i$ เป็น{\wbr}ผลลัพธ์\\
\hspace*{0.2in} ตอบ{\wbr}ว่า{\wbr}ไม่{\wbr}พบ{\wbr}ค่า{\wbr}ที่{\wbr}ต้องการ{\wbr}
\end{algt}

ใน{\wbr}การ{\wbr}พัฒนา{\wbr}โปรแกรม{\wbr}จริง ๆ
เรา{\wbr}จะ{\wbr}ต้อง{\wbr}จัดการ{\wbr}ใน{\wbr}กรณี{\wbr}ที่{\wbr}จะ{\wbr}ต้อง{\wbr}ตอบ{\wbr}ว่า{\wbr}ไม่{\wbr}พบ{\wbr}ค่า{\wbr}ที่{\wbr}ต้องการ{\wbr}ให้{\wbr}ชัดเจน{\wbr}กว่า{\wbr}นี้{\wbr}
แต่{\wbr}ใน{\wbr}ขณะนี้{\wbr}เรา{\wbr}จะ{\wbr}สมมติ{\wbr}ว่า{\wbr}โปรแกรมย่อย{\wbr}สามารถ{\wbr}ตอบ{\wbr}แบบ{\wbr}นี้{\wbr}ได้{\wbr}

\begin{quiz}{}
ใน{\wbr}กรณี{\wbr}ของ{\wbr}โปรแกรมย่อย{\wbr}สำหรับ{\wbr}หา{\wbr}ผลรวม เรา{\wbr}พบ{\wbr}ว่า{\wbr}โปรแกรม{\wbr}ทำงาน{\wbr}ใน{\wbr}เวลา{\wbr}ที่{\wbr}แปรผัน{\wbr}กับ{\wbr}ค่า $n$
เสมอ เป็น{\wbr}ไป{\wbr}ได้{\wbr}หรือ{\wbr}ไม่ ที่{\wbr}โปรแกรมย่อย{\wbr}สำหรับ{\wbr}จะ{\wbr}ทำงาน{\wbr}โดย{\wbr}วน{\wbr}รอบ{\wbr}เป็น{\wbr}จำนวน{\wbr}ครั้ง{\wbr}ที่{\wbr}น้อย{\wbr}กว่า{\wbr}ค่า $n$ มาก? และ{\wbr}เป็น{\wbr}ใน{\wbr}กรณี{\wbr}ใด?
\end{quiz}

\begin{quiz}{}
สำหรับ{\wbr}อาร์เรย์{\wbr}ที่{\wbr}มี{\wbr}ข้อมูล $n$ ตัว เมื่อใด{\wbr}ที่{\wbr}โปรแกรมย่อย{\wbr}จะ{\wbr}ทำงาน{\wbr}โดย{\wbr}วน{\wbr}รอบ{\wbr}มาก{\wbr}ที่สุด{\wbr}
\end{quiz}

โปรแกรมย่อย{\wbr}ข้างต้น{\wbr}อาจ{\wbr}จะ{\wbr}ทำงาน{\wbr}ได้{\wbr}รวดเร็ว{\wbr}มาก ถ้า{\wbr}ข้อมูล{\wbr}ที่{\wbr}ต้องการ{\wbr}ค้นหา{\wbr}อยู่{\wbr}ตอน{\wbr}ต้น{\wbr}ของ{\wbr}อาร์เรย์
โปรแกรมย่อย{\wbr}ลักษณะ{\wbr}นี้{\wbr}เป็น{\wbr}ตัวอย่าง{\wbr}ที่{\wbr}ดี{\wbr}ของ{\wbr}โปรแกรมย่อย{\wbr}ที่{\wbr}เวลา{\wbr}การ{\wbr}ทำงาน{\wbr}ขึ้น{\wbr}กับ{\wbr}ข้อมูล{\wbr}ป้อน{\wbr}เข้า{\wbr}
ทำ{\wbr}ให้{\wbr}ใน{\wbr}การ{\wbr}วิเคราะห์{\wbr}เวลา{\wbr}การ{\wbr}ทำงาน{\wbr}นั้น เรา{\wbr}จำเป็น{\wbr}จะ{\wbr}ต้อง{\wbr}พิจารณา{\wbr}ข้อมูล{\wbr}ป้อน{\wbr}เข้า{\wbr}ด้วย{\wbr}
อย่างไรก็ตาม{\wbr}เรา{\wbr}ไม่{\wbr}สามารถ{\wbr}ที่{\wbr}จะ{\wbr}วิเคราะห์{\wbr}เวลา{\wbr}การ{\wbr}ทำงาน{\wbr}ของ{\wbr}โปรแกรม{\wbr}ลำ{\wbr}ลอง{\wbr}บน{\wbr}ข้อมูล{\wbr}ป้อน{\wbr}เข้า{\wbr}ทุก{\wbr}รูปแบบ{\wbr}ได้{\wbr}
เพราะว่า{\wbr}จำนวน{\wbr}ของ{\wbr}ข้อมูล{\wbr}ป้อน{\wbr}เข้า{\wbr}นั้น{\wbr}มี{\wbr}ไม่{\wbr}จำกัด{\wbr}

ใน{\wbr}ทาง{\wbr}ปฏิบัติ{\wbr}แล้ว เรา{\wbr}จึง{\wbr}จะ{\wbr}แบ่ง{\wbr}วิเคราะห์{\wbr}เวลา{\wbr}การ{\wbr}ทำงาน{\wbr}เป็น{\wbr}กรณี{\wbr}ย่อย ๆ สาม{\wbr}กรณี{\wbr}คือ{\wbr}
\begin{itemize}
\item การ{\wbr}วิเคราะห์{\wbr}ใน{\wbr}กรณี{\wbr}ที่{\wbr}ดี{\wbr}ที่สุด (best-case analysis),
\item การ{\wbr}วิเคราะห์{\wbr}ใน{\wbr}กรณี{\wbr}ที่{\wbr}เลวร้าย{\wbr}ที่สุด (worst-case analysis), และ{\wbr}
\item การ{\wbr}วิเคราะห์{\wbr}ใน{\wbr}กรณี{\wbr}เฉลี่ย (average-case analysis)
\end{itemize}

สำหรับ{\wbr}การ{\wbr}วิเคราะห์{\wbr}ใน{\wbr}กรณี{\wbr}เฉลี่ย{\wbr}นั้น เป็น{\wbr}การ{\wbr}วิเคราะห์{\wbr}เชิง{\wbr}ความน่าจะเป็น{\wbr}
เรา{\wbr}จำเป็น{\wbr}จะ{\wbr}ต้อง{\wbr}นิยาม{\wbr}ลักษณะ{\wbr}การ{\wbr}กระจาย{\wbr}ของ{\wbr}ข้อมูล{\wbr}ป้อน{\wbr}เข้า{\wbr}ให้{\wbr}ชัดเจน จึง{\wbr}จะ{\wbr}สามารถ{\wbr}กระทำ{\wbr}ได้{\wbr}
เรา{\wbr}จะ{\wbr}ได้{\wbr}ศึกษา{\wbr}ตัวอย่าง{\wbr}การ{\wbr}วิเคราะห์{\wbr}นี้{\wbr}ใน{\wbr}บท{\wbr}ที่~\ref{chapter:randomization}
ใน{\wbr}ที่นี้{\wbr}เรา{\wbr}จะ{\wbr}สนใจ{\wbr}เฉพาะ{\wbr}การ{\wbr}วิเคราะห์{\wbr}กรณี{\wbr}ที่{\wbr}ดี{\wbr}ที่สุด และ{\wbr}การ{\wbr}วิเคราะห์{\wbr}ใน{\wbr}กรณี{\wbr}ที่{\wbr}เลวร้าย{\wbr}ที่สุด{\wbr}

กรณี{\wbr}ที่{\wbr}ดี{\wbr}ที่สุด{\wbr}คือ{\wbr}กรณี{\wbr}ที่{\wbr}มี{\wbr}การ{\wbr}วน{\wbr}รอบ{\wbr}เพียง{\wbr}รอบ{\wbr}เดียว นั้น{\wbr}คือ{\wbr}เป็น{\wbr}กรณี{\wbr}ที่ $A[0] = x$
สังเกต{\wbr}ว่า{\wbr}ถ้า{\wbr}เรา{\wbr}สมมติ{\wbr}ให้การ{\wbr}ประมวลผล{\wbr}แต่ละ{\wbr}บรรทัด{\wbr}ใช้เวลา 1 หน่วย ใน{\wbr}กรณี{\wbr}ที่{\wbr}ดี{\wbr}ที่สุด{\wbr}
โปรแกรม{\wbr}ลำ{\wbr}ลอง{\wbr}ดังกล่าว{\wbr}จะ{\wbr}ใช้เวลา{\wbr}ทำงาน $4$ หน่วย{\wbr}

กรณี{\wbr}ที่{\wbr}เลวร้าย{\wbr}ที่สุด{\wbr}เกิด{\wbr}ขึ้น{\wbr}เมื่อ{\wbr}ไม่{\wbr}พบ{\wbr}ข้อมูล{\wbr}ที่{\wbr}ต้องการ{\wbr}หา{\wbr}
สังเกต{\wbr}ว่า{\wbr}โปรแกรม{\wbr}จะ{\wbr}ทำงาน{\wbr}วน{\wbr}อยู่{\wbr}ที่{\wbr}สอง{\wbr}บรรทัด{\wbr}แรก{\wbr}เป็น{\wbr}จำนวน $n$ ครั้ง และ{\wbr}คืนคำ{\wbr}ตอบ{\wbr}
ดังนั้น{\wbr}โปรแกรม{\wbr}จะ{\wbr}ใช้เวลา{\wbr}ทำงาน $2n + 1$ หน่วย{\wbr}

\subsubsection{การ{\wbr}เพิ่ม{\wbr}ข้อมูล{\wbr}ลง{\wbr}ไป{\wbr}ท้าย{\wbr}รายการ}

เรา{\wbr}จะ{\wbr}เพิ่ม{\wbr}ข้อมูล{\wbr}ลง{\wbr}ไป{\wbr}ตอน{\wbr}ท้าย{\wbr}ของ{\wbr}ข้อมูล{\wbr}ใน{\wbr}อาร์เรย์
นั่น{\wbr}คือ{\wbr}ใส่{\wbr}ข้อมูล{\wbr}ใน{\wbr}อาร์เรย์{\wbr}ที่{\wbr}มี{\wbr}ดัชนี{\wbr}มาก{\wbr}กว่า{\wbr}ดัชนี{\wbr}ตัว{\wbr}สุดท้าย{\wbr}
โปรแกรม{\wbr}ลำ{\wbr}ลอง{\wbr}ที่{\wbr}น่าจะ{\wbr}ทำงาน{\wbr}ได้{\wbr}เขียน{\wbr}ดังนี้{\wbr}

\begin{algt}
\noindent {\bf เพิ่ม{\wbr}ข้อมูล $x$ ใน{\wbr}ตอน{\wbr}ท้าย{\wbr}อาร์เรย์ $A$ ที่{\wbr}มี{\wbr}ข้อมูล $n$ ตัว}\\
\hspace*{0.2in} $A[n] \leftarrow x$\\
\hspace*{0.2in} $n \leftarrow n + 1$
\end{algt}

อย่างไรก็ตาม ใน{\wbr}การ{\wbr}นำ{\wbr}ไป{\wbr}ใช้{\wbr}จริง โปรแกรม{\wbr}ลำ{\wbr}ลอง{\wbr}ดังกล่าว{\wbr}อาจ{\wbr}จะ{\wbr}ทำ{\wbr}ให้{\wbr}เกิด{\wbr}ข้อผิดพลาด{\wbr}ขึ้น{\wbr}ระหว่าง{\wbr}การ{\wbr}ทำงาน{\wbr}ได้{\wbr}

\begin{quiz}{}
กรณี{\wbr}ใด{\wbr}ที่{\wbr}โปรแกรม{\wbr}ลำ{\wbr}ลอง{\wbr}ข้างต้น{\wbr}อาจ{\wbr}ทำ{\wbr}ให้{\wbr}เกิด{\wbr}ข้อผิดพลาด{\wbr}ขึ้น{\wbr}ระหว่าง{\wbr}การ{\wbr}ทำงาน{\wbr}
\end{quiz}
\begin{quizans}
จาก{\wbr}ที่{\wbr}เรา{\wbr}ได้{\wbr}เคย{\wbr}เกริ่น{\wbr}บ้าง{\wbr}แล้ว{\wbr}ว่า ใน{\wbr}การ{\wbr}ใช้{\wbr}งาน{\wbr}อาร์เรย์
โดยมาก{\wbr}จะ{\wbr}ต้อง{\wbr}ระบุ{\wbr}ขอบเขต{\wbr}หรือ{\wbr}จำนวน{\wbr}ข้อมูล{\wbr}มาก{\wbr}ที่สุด{\wbr}ที่{\wbr}เก็บ{\wbr}ใน{\wbr}อาร์เรย์{\wbr}ได้{\wbr}
ใน{\wbr}กรณี{\wbr}ของ{\wbr}โปรแกรม{\wbr}ลำ{\wbr}ลอง{\wbr}นี้{\wbr}ถ้า{\wbr}เรา{\wbr}เรียก{\wbr}ใช้{\wbr}เมื่อ $n$
มี{\wbr}ขนาด{\wbr}มาก{\wbr}กว่า{\wbr}หรือ{\wbr}เท่า{\wbr}กับ{\wbr}จำนวน{\wbr}ข้อมูล{\wbr}ที่{\wbr}อาร์เรย์{\wbr}เก็บ{\wbr}ได้ คำสั่ง $A[n]\leftarrow x$
ก็{\wbr}อาจ{\wbr}จะ{\wbr}เขียน{\wbr}ข้อมูล{\wbr}ลง{\wbr}ใน{\wbr}หน่วยความจำ{\wbr}บริเวณ{\wbr}ที่อยู่{\wbr}นอก{\wbr}ขอบเขต{\wbr}ของ{\wbr}อาร์เรย์ $A$ ได้{\wbr}
\end{quizans}

ดังนั้น{\wbr}เพื่อ{\wbr}ความ{\wbr}ไม่{\wbr}ประมาท{\wbr}
โปรแกรมย่อย{\wbr}ควร{\wbr}จะ{\wbr}ต้อง{\wbr}ตรวจสอบ{\wbr}ขนาด{\wbr}ของ{\wbr}อาร์เรย์{\wbr}เพื่อ{\wbr}ป้องกัน{\wbr}ความผิด{\wbr}พลาด{\wbr}นี้{\wbr}ด้วย{\wbr}
ใน{\wbr}การ{\wbr}เขียน{\wbr}ต่อไป{\wbr}เรา{\wbr}จะ{\wbr}ให้ $MAXLEN$ เป็น{\wbr}ค่าคงที่{\wbr}แทน{\wbr}ขนาด{\wbr}มาก{\wbr}ที่สุด{\wbr}ของ{\wbr}อาร์เรย์ $A$
เรา{\wbr}ปรับ{\wbr}แก้{\wbr}โปรแกรมย่อย{\wbr}ได้{\wbr}ดัง{\wbr}ด้าน{\wbr}ล่าง{\wbr}

\begin{algt}
\noindent {\bf เพิ่ม{\wbr}ข้อมูล $x$ ใน{\wbr}ตอน{\wbr}ท้าย{\wbr}อาร์เรย์ $A$ ที่{\wbr}มี{\wbr}ข้อมูล $n$ ตัว (แก้ไข)}\\
\hspace*{0.2in} ถ้า $n < MAXLEN$ แล้ว\\
\hspace*{0.2in}\hspace*{0.2in} $A[n] \leftarrow x$\\
\hspace*{0.2in}\hspace*{0.2in} $n \leftarrow n + 1$\\
\hspace*{0.2in} ไม่{\wbr}เช่นนั้น\\
\hspace*{0.2in}\hspace*{0.2in} รายงาน{\wbr}ว่า{\wbr}ไม่{\wbr}สามารถ{\wbr}เพิ่ม{\wbr}ข้อมูล{\wbr}ได้{\wbr}
\end{algt}

โปรแกรมย่อย{\wbr}นี้ ใน{\wbr}การ{\wbr}วิเคราะห์{\wbr}เวลา{\wbr}การ{\wbr}ทำงาน{\wbr}มี{\wbr}สอง{\wbr}กรณี{\wbr}ให้{\wbr}เรา{\wbr}พิจารณา สังเกต{\wbr}ว่า{\wbr}
จะ{\wbr}ใช้เวลา{\wbr}ใน{\wbr}การ{\wbr}ทำงาน{\wbr}ไม่{\wbr}เกิน $3$ หน่วย{\wbr}ไม่ว่า{\wbr}ใน{\wbr}กรณี{\wbr}ใด 

ใน{\wbr}กรณี{\wbr}แรก (กรณี{\wbr}ที่ $n<MAXLEN$) โปรแกรมย่อย{\wbr}จะ{\wbr}ใช้เวลา{\wbr}การ{\wbr}ทำงาน $3$ หน่วย{\wbr}
และ{\wbr}ใน{\wbr}อีก{\wbr}กรณี{\wbr}จะ{\wbr}ใช้เวลา{\wbr}การ{\wbr}ทำงาน $2$ หน่วย อย่างไรก็ตาม{\wbr}
ผู้อ่าน{\wbr}อย่า{\wbr}เพิ่ง{\wbr}รีบ{\wbr}สรุป{\wbr}ว่า{\wbr}กรณี{\wbr}แรก{\wbr}ทำงาน{\wbr}จริง ๆ ได้{\wbr}เร็ว{\wbr}กว่า เพราะว่า{\wbr}ความ{\wbr}แตกต่าง{\wbr}นี้{\wbr}จริง ๆ
แล้ว{\wbr}เกิด{\wbr}จาก{\wbr}ข้อสมมติ{\wbr}ว่า{\wbr}การ{\wbr}ทำงาน{\wbr}ใน{\wbr}ทุก{\wbr}คำสั่ง{\wbr}มี{\wbr}ความ{\wbr}เร็ว{\wbr}เท่า{\wbr}กัน{\wbr}คือ 1 หน่วย{\wbr}
ดังนั้น{\wbr}ประเด็น{\wbr}สำคัญ{\wbr}ของ{\wbr}การ{\wbr}วิเคราะห์{\wbr}นี้{\wbr}คือ{\wbr}โปรแกรมย่อย{\wbr}นี้{\wbr}ทำงาน{\wbr}ใน{\wbr}เวลา{\wbr}ที่{\wbr}ไม่{\wbr}ขึ้น{\wbr}กับ{\wbr}ค่า $n$

\subsubsection{การ{\wbr}ลบ{\wbr}ข้อมูล{\wbr}ใน{\wbr}รายการ{\wbr}และ{\wbr}การ{\wbr}แทรก{\wbr}ข้อมูล{\wbr}ใน{\wbr}รายการ}

สำหรับ{\wbr}การ{\wbr}ลบ{\wbr}ข้อมูล{\wbr}และ{\wbr}แทรก{\wbr}ข้อมูล{\wbr}ใน{\wbr}รายการ{\wbr}
โปรแกรมย่อย{\wbr}ที่{\wbr}ประมวลผล{\wbr}นั้น{\wbr}จะ{\wbr}รับ{\wbr}ดัชนี{\wbr}ของ{\wbr}ข้อมูล{\wbr}ที่{\wbr}ต้องการ{\wbr}ลบ{\wbr}
และ{\wbr}ดัชนี{\wbr}ที่{\wbr}ต้องการ{\wbr}ให้{\wbr}นำ{\wbr}ข้อมูล{\wbr}ไป{\wbr}แทรก{\wbr}ต่อ{\wbr}จาก{\wbr}ตำแหน่ง{\wbr}นั้น{\wbr}
ถ้า{\wbr}อาร์เรย์{\wbr}เริ่มต้น{\wbr}ของ{\wbr}เรา{\wbr}มี{\wbr}ข้อมูล 9 ตัว ดัง{\wbr}ด้าน{\wbr}ล่าง{\wbr}

\begin{center}
\begin{tabular}{|c|c|c|c|c|c|c|c|c|c|c|}
\hline
ดัชนี & 0 & 1 & 2 & 3 & 4 & 5 & 6 & 7 & 8 & 9\\
\hline
ข้อมูล & 2 & 3 & 5 & 7 & 11 & 13 & 17 & 19 & 23 & ?\\
\hline
\end{tabular}

จำนวน{\wbr}ข้อมูล $n = 9$
\end{center}

สังเกต{\wbr}ว่า{\wbr}เรา{\wbr}ละ{\wbr}ข้อมูล{\wbr}ที่{\wbr}ใน{\wbr}อาร์เรย์{\wbr}ที่{\wbr}มี{\wbr}ดัชนี{\wbr}อยู่{\wbr}นอก{\wbr}ขอบเขต{\wbr}ของ{\wbr}ข้อมูล{\wbr}ใน{\wbr}รายการ{\wbr}ไป{\wbr}
(โดย{\wbr}แสดง{\wbr}ด้วย{\wbr}เครื่องหมาย ?)

การ{\wbr}ลบ{\wbr}ข้อมูล{\wbr}ที่{\wbr}มี{\wbr}ดัชนี{\wbr}เป็น 3 ให้{\wbr}ผล{\wbr}ดังนี้{\wbr}

\begin{center}
\begin{tabular}{|c|c|c|c|c|c|c|c|c|c|c|}
\hline
ดัชนี & 0 & 1 & 2 & 3 & 4 & 5 & 6 & 7 & 8 & 9\\
\hline
ข้อมูล & 2 & 3 & 5 & 11 & 13 & 17 & 19 & 23 & ? & ? \\
\hline
\end{tabular}

จำนวน{\wbr}ข้อมูล $n = 8$
\end{center}

จาก{\wbr}อาร์เรย์{\wbr}ดังกล่าว{\wbr}ส การ{\wbr}แทรก{\wbr}ข้อมูล 99 เข้า{\wbr}ไป{\wbr}หลัง{\wbr}ข้อมูล{\wbr}ที่{\wbr}มี{\wbr}ดัชนี{\wbr}เป็น 1 ให้{\wbr}ผล{\wbr}ดังนี้{\wbr}

\begin{center}
\begin{tabular}{|c|c|c|c|c|c|c|c|c|c|c|}
\hline
ดัชนี & 0 & 1 & 2 & 3 & 4 & 5 & 6 & 7 & 8 & 9\\
\hline
ข้อมูล & 2 & 3 & 99 & 5 & 11 & 13 & 17 & 19 & 23 & ? \\
\hline
\end{tabular}

จำนวน{\wbr}ข้อมูล $n = 9$
\end{center}

\begin{quiz}{}
การ{\wbr}ประมวลผล{\wbr}ทั้ง{\wbr}สอง{\wbr}แบบ{\wbr}มี{\wbr}กระบวนการ{\wbr}หนึ่ง{\wbr}ที่{\wbr}ต้อง{\wbr}ดำเนินการ{\wbr}คล้าย ๆ กัน คือ{\wbr}อะไร{\wbr}
\end{quiz}

การ{\wbr}ประมวลผล{\wbr}ทั้ง{\wbr}สอง{\wbr}แบบ{\wbr}นี้{\wbr}แสดง{\wbr}ให้{\wbr}เห็น{\wbr}ข้อจำกัด{\wbr}ของ{\wbr}การ{\wbr}เก็บ{\wbr}ข้อมูล{\wbr}แบบ{\wbr}รายการ{\wbr}ด้วย{\wbr}อาร์เรย์
(ยกเว้น{\wbr}จะ{\wbr}มี{\wbr}เทคนิค{\wbr}พิเศษ{\wbr}อื่น ๆ ประกอบ)
ที่{\wbr}โครงสร้าง{\wbr}ข้อมูล{\wbr}เช่น{\wbr}ลิงก์ลิสต์{\wbr}ที่{\wbr}เรา{\wbr}จะ{\wbr}พิจารณา{\wbr}ใน{\wbr}บท{\wbr}ที่~\ref{chapter:linked-lists}
สามารถ{\wbr}จัดการ{\wbr}ได้{\wbr}เป็น{\wbr}อย่าง{\wbr}ดี{\wbr}

โปรแกรม{\wbr}ลำ{\wbr}ลอง{\wbr}ด้าน{\wbr}ล่าง{\wbr}แสดง{\wbr}การ{\wbr}ลบ{\wbr}ข้อมูล{\wbr}ที่{\wbr}มี{\wbr}ดัชนี $i$ ใน{\wbr}รายการ{\wbr}ใน{\wbr}อาร์เรย์

\begin{algt}
\label{algo:array-deletion1}
\noindent {\bf ลบ{\wbr}ข้อมูล{\wbr}ที่{\wbr}มี{\wbr}ดัชนี $i$ ใน{\wbr}รายการ{\wbr}ที่{\wbr}เก็บ{\wbr}ใน{\wbr}อาร์เรย์ $A$ ที่{\wbr}มี{\wbr}ขนาด $n$ (แก้ไข)}\\
\hspace*{0.2in} พิจารณา{\wbr}ให้ ตัวแปร $i\leftarrow i,i+1,\ldots,n-1$\\
\hspace*{0.2in}\hspace*{0.2in} $A[i]\leftarrow A[i+1]$\\
\hspace*{0.2in} $n\leftarrow n-1$
\end{algt}

\begin{quiz}{}
โปรแกรม{\wbr}ลำ{\wbr}ลอง~\ref{algo:array-deletion1}
อาจ{\wbr}ทำ{\wbr}ให้{\wbr}เกิด{\wbr}ความผิด{\wbr}พลาด{\wbr}ระหว่าง{\wbr}การ{\wbr}ทำงาน{\wbr}ได้{\wbr}ใน{\wbr}บาง{\wbr}กรณี{\wbr}เพราะว่า{\wbr}ไม่{\wbr}ได้{\wbr}ตรวจสอบ{\wbr}เงื่อนไข{\wbr}บาง{\wbr}อย่าง{\wbr}
เงื่อนไข{\wbr}นั้น{\wbr}คือ{\wbr}อะไร?
\end{quiz}

สังเกต{\wbr}ว่า{\wbr}ถ้า{\wbr}ดัชนี{\wbr}อยู่{\wbr}นอก{\wbr}ของ{\wbr}เขต{\wbr}ของ{\wbr}รายการ{\wbr}
หรือ{\wbr}ใน{\wbr}กรณี{\wbr}ที่{\wbr}ไม่{\wbr}มี{\wbr}ข้อมูล{\wbr}โปรแกรม{\wbr}ลำ{\wbr}ลอง{\wbr}อาจ{\wbr}เกิด{\wbr}ปัญหา{\wbr}ระหว่าง{\wbr}ทำงาน{\wbr}
โปรแกรม{\wbr}ลำ{\wbr}ลอง{\wbr}ด้าน{\wbr}ล่าง{\wbr}เพิ่ม{\wbr}เงื่อนไข{\wbr}ใน{\wbr}การ{\wbr}ตรวจสอบ{\wbr}นี้ สำหรับ{\wbr}ใน{\wbr}กรณี{\wbr}เช่น{\wbr}ใน{\wbr}ตัวอย่าง{\wbr}นี้ ภาษา{\wbr}
C++ จะ{\wbr}มี{\wbr}วิธีการ{\wbr}จัดการ{\wbr}รายงาน{\wbr}ความผิด{\wbr}พลาด{\wbr}กลับ{\wbr}ไป{\wbr}ยัง{\wbr}โปรแกรมหลัก{\wbr}ที่{\wbr}เรียก{\wbr}ใช้{\wbr}อย่าง{\wbr}เป็น{\wbr}ระบบ{\wbr}
โดย{\wbr}ใช้{\wbr}แนว{\wbr}คิด{\wbr}ที่{\wbr}เรียก{\wbr}ว่า exception ซึ่ง{\wbr}เรา{\wbr}จะ{\wbr}ได้{\wbr}พิจารณา{\wbr}ต่อไป{\wbr}ใน{\wbr}บท{\wbr}ที่ XXXX (TODO:
เพิ่ม{\wbr}หรือ{\wbr}ลบ)

\begin{algt}
\label{algo:array-deletion2}
\noindent {\bf ลบ{\wbr}ข้อมูล{\wbr}ที่{\wbr}มี{\wbr}ดัชนี $i$ ใน{\wbr}รายการ{\wbr}ที่{\wbr}เก็บ{\wbr}ใน{\wbr}อาร์เรย์ $A$ ที่{\wbr}มี{\wbr}ขนาด $n$ (แก้ไข)}\\
\hspace*{0.2in} ถ้า $i > n-1$ หรือ $i < 0$\\
\hspace*{0.2in}\hspace*{0.2in} รายงาน{\wbr}ความผิด{\wbr}พลาด{\wbr}ว่า{\wbr}ดัชนี{\wbr}อยู่{\wbr}นอก{\wbr}ขอบเขต แล้ว{\wbr}จบ{\wbr}การ{\wbr}ทำงาน\\
\hspace*{0.2in} พิจารณา{\wbr}ให้ ตัวแปร $i\leftarrow i,i+1,\ldots,n-1$\\
\hspace*{0.2in}\hspace*{0.2in} $A[i]\leftarrow A[i+1]$\\
\hspace*{0.2in} $n\leftarrow n-1$
\end{algt}

สังเกต{\wbr}ว่า{\wbr}เวลา{\wbr}ที่{\wbr}โปรแกรม{\wbr}ลำ{\wbr}ลอง{\wbr}ข้างต้น{\wbr}ใช้{\wbr}การ{\wbr}ทำงาน ขึ้น{\wbr}กับ{\wbr}ตำแหน่ง{\wbr}ของ{\wbr}ข้อมูล{\wbr}
เช่นเดียวกับ{\wbr}การ{\wbr}วิเคราะห์{\wbr}ใน{\wbr}ส่วน{\wbr}ของ{\wbr}การ{\wbr}ค้น{\wbr}ข้อมูล เรา{\wbr}สามารถ{\wbr}พิจารณา{\wbr}กรณี{\wbr}ต่าง ๆ ได้{\wbr}สาม{\wbr}แบบ{\wbr}

\begin{quiz}{}
กรณี{\wbr}ใด{\wbr}ที่{\wbr}ทำ{\wbr}ให้{\wbr}โปรแกรม{\wbr}ลำ{\wbr}ลอง~\ref{algo:array-deletion2}
ใช้เวลา{\wbr}ใน{\wbr}การ{\wbr}ทำงาน{\wbr}น้อย{\wbr}ที่สุด (best case) และ{\wbr}ใช้เวลา{\wbr}เป็น{\wbr}เท่าใด?
\end{quiz}

\begin{quiz}{}
กรณี{\wbr}ใด{\wbr}ที่{\wbr}ทำ{\wbr}ให้{\wbr}โปรแกรม{\wbr}ลำ{\wbr}ลอง{\wbr}ทำงาน{\wbr}โดย{\wbr}ใช้เวลา{\wbr}มาก{\wbr}ที่สุด (worst case)  และ{\wbr}ใช้เวลา{\wbr}เป็น{\wbr}เท่าใด{\wbr}
\end{quiz}

เรา{\wbr}จะ{\wbr}ละ{\wbr}การ{\wbr}เขียน{\wbr}โปรแกรม{\wbr}ลำ{\wbr}ลองของ{\wbr}การ{\wbr}แทรก{\wbr}ข้อมูล{\wbr}ใน{\wbr}รายการ{\wbr}ไว้{\wbr}เป็น{\wbr}แบบฝึกหัด{\wbr}ท้าย{\wbr}บท{\wbr}

\section{การ{\wbr}วิเคราะห์{\wbr}เชิง{\wbr}เส้น{\wbr}กำกับ}

\section{การ{\wbr}ประกาศ{\wbr}และ{\wbr}ใช้{\wbr}งาน{\wbr}อาร์เรย์{\wbr}ใน{\wbr}ภาษา C/C++}

ใน{\wbr}ภาษา C หรือ C++ เรา{\wbr}สามารถ{\wbr}ประกาศ{\wbr}ตัวแปร{\wbr}ประเภท{\wbr}อาร์เรย์{\wbr}ได้{\wbr}โดย{\wbr}ใช้{\wbr}รูปแบบ{\wbr}ดังนี้{\wbr}

\begin{center}
ชนิด{\wbr}ข้อมูล ชื่อ{\wbr}ตัวแปร[ขนาด];
\end{center}

ตัวอย่าง{\wbr}ด้าน{\wbr}ล่าง{\wbr}แสดง{\wbr}การ{\wbr}ประกาศ{\wbr}ตัวแปร{\wbr}อาร์เรย์{\wbr}ของ{\wbr}จำนวนเต็ม ({\ttt int})
และ{\wbr}อาร์เรย์{\wbr}ของ{\wbr}อักขระ ({\ttt char})

\latintext
\begin{codelist}{C++}{}
int a[100];
char buf[1000];
\end{codelist}
\thaitext

\latintext
\begin{codelist}{C++}{caption={\thaitext ทดสอบ\latintext},label=code:array-def}
int a[100];
char buf[1000];
\end{codelist}
\thaitext

ดู{\wbr}ตัวอย่าง{\wbr}ที่~\ref{code:array-def}

\section{พอยน์เตอร์และ{\wbr}การ{\wbr}ใช้{\wbr}งาน{\wbr}ใน{\wbr}ภาษา C/C++}
\label{sect:array-pointer-c}

เนื่องจาก{\wbr}พอยน์เตอร์{\wbr}เป็น{\wbr}ชนิด{\wbr}ข้อมูล{\wbr}ที่{\wbr}ใกล้ชิด{\wbr}กับ{\wbr}สถาปัตยกรรม{\wbr}ของ{\wbr}คอมพิวเตอร์{\wbr}มาก{\wbr}ที่สุด{\wbr}ใน{\wbr}ภาษา{\wbr}
C/C++ เรา{\wbr}จะ{\wbr}เริ่ม{\wbr}โดย{\wbr}ศึกษา{\wbr}โครงสร้าง{\wbr}การ{\wbr}เก็บ{\wbr}ข้อมูล{\wbr}ใน{\wbr}หน่วยความจำ{\wbr}คอมพิวเตอร์{\wbr}กัน{\wbr}ก่อน{\wbr}

คอมพิวเตอร์{\wbr}เก็บ{\wbr}ข้อมูล{\wbr}แบบ{\wbr}ดิจิทัล หน่วย{\wbr}ย่อย{\wbr}ที่สุด{\wbr}ของ{\wbr}ข้อมูล{\wbr}แบบ{\wbr}ดิจิทัล{\wbr}คือ{\wbr}บิต (bit ย่อ{\wbr}มา{\wbr}จาก{\wbr}
binary digit) ซึ่ง{\wbr}เป็น{\wbr}ข้อมูล{\wbr}ที่{\wbr}มี{\wbr}สอง{\wbr}สถานะ โดยมาก{\wbr}จะ{\wbr}แทน{\wbr}ด้วย 0 และ 1 หรือ{\wbr}ปิด{\wbr}กับ{\wbr}เปิด{\wbr}
แต่{\wbr}อาจ{\wbr}จะ{\wbr}เป็น{\wbr}อย่าง{\wbr}อื่น{\wbr}ก็ได้{\wbr}เช่น มี{\wbr}แสง ไม่{\wbr}มี{\wbr}แสง, ศักย์{\wbr}ไฟฟ้า{\wbr}สูง ศักย์{\wbr}ไฟฟ้า{\wbr}ต่ำ{\wbr}เป็นต้น{\wbr}

หน่วยความจำ{\wbr}คอมพิวเตอร์{\wbr}ใน{\wbr}ปัจจุบัน{\wbr}เก็บ{\wbr}ข้อมูล{\wbr}หลาย{\wbr}ล้าน{\wbr}บิต{\wbr}
การ{\wbr}อ้าง{\wbr}ถึง{\wbr}ข้อมูล{\wbr}จำนวน{\wbr}มาก{\wbr}เหล่านั้น{\wbr}กระทำ{\wbr}ผ่าน{\wbr}ทาง{\wbr}ระบบ{\wbr}เรียก{\wbr}ที่อยู่ (address)
กล่าวคือ{\wbr}ข้อมูล{\wbr}ที่{\wbr}อ้าง{\wbr}ถึง{\wbr}ได้{\wbr}ทุก{\wbr}หน่วย{\wbr}จะ{\wbr}มี{\wbr}ที่อยู่{\wbr}เฉพาะ{\wbr}ใช้{\wbr}สำหรับ{\wbr}อ้าง{\wbr}ถึง{\wbr}
ลักษณะ{\wbr}การ{\wbr}อ้าง{\wbr}ถึง{\wbr}ข้อมูล{\wbr}เช่นนี้{\wbr}ก็{\wbr}ไม่{\wbr}ต่าง{\wbr}จาก{\wbr}การ{\wbr}ที่{\wbr}เรา{\wbr}ใช้{\wbr}ดัชนี{\wbr}ใน{\wbr}การ{\wbr}อ้าง{\wbr}ถึง{\wbr}ข้อมูล{\wbr}ใน{\wbr}อาร์เรย์{\wbr}นั่นเอง{\wbr}

อย่างไรก็ตาม ใน{\wbr}การ{\wbr}กำหนด{\wbr}ที่อยู่{\wbr}นั้น จะ{\wbr}ไม่{\wbr}ได้{\wbr}กำหนด{\wbr}ให้{\wbr}กับ{\wbr}ข้อมูล{\wbr}ทุก{\wbr}บิต{\wbr}โดย{\wbr}ตรง{\wbr}
เนื่องจาก{\wbr}บิต{\wbr}เป็น{\wbr}หน่วย{\wbr}ที่{\wbr}เล็ก{\wbr}เกิน{\wbr}ไป{\wbr}ใน{\wbr}หลาย ๆ กรณี แต่{\wbr}จะ{\wbr}กำหนด{\wbr}ให้{\wbr}กับ{\wbr}กลุ่ม{\wbr}ของ{\wbr}บิต{\wbr}ที่{\wbr}เรียง{\wbr}กัน{\wbr}
กลุ่ม{\wbr}ละ 8 บิต ซึ่ง{\wbr}เรียก{\wbr}ว่า{\wbr}ไบท์ (byte) ซึ่ง{\wbr}สามารถ{\wbr}พิจารณา{\wbr}เป็น{\wbr}เลขฐานสอง 8 หลัก{\wbr}ก็ได้{\wbr}
ตัวอย่าง{\wbr}ของ{\wbr}การ{\wbr}เก็บ{\wbr}ข้อมูล{\wbr}แสดง{\wbr}ดัง{\wbr}รูป{\wbr}ที่~\ref{fig:array-memory-as-array}

\begin{figure}
\begin{center}\begin{tabular}{|c|c|c|}
\hline
\hline
ตำแหน่ง & ข้อมูล{\wbr}เป็น{\wbr}บิต & ข้อมูล{\wbr}ถ้า{\wbr}พิจารณา{\wbr}เป็น{\wbr}ตัวเลข\\
\hline
\hline
0 & 0000 0000 & 0 \\
\hline
1 & 0000 1111 & 15 \\
\hline
$\vdots$ & $\vdots$ & $\vdots$ \\
\hline
1000000 & 1001 0010 & 146 \\
\hline
1000001 & 1111 1111 & 255 \\
\hline
1000002 & 0001 0000 & 16 \\
\hline
1000003 & 1010 1010 & 170 \\
\hline
$\vdots$ & $\vdots$ & $\vdots$ \\
\hline
\end{tabular}\end{center}
\caption{ตัวอย่าง{\wbr}การ{\wbr}เก็บ{\wbr}ข้อมูล{\wbr}ใน{\wbr}หน่วยความจำ{\wbr}พร้อมด้วย{\wbr}ที่อยู่}
\label{fig:array-memory-as-array}
\end{figure}

ข้อมูล{\wbr}ที่{\wbr}เก็บ{\wbr}ใน{\wbr}หน่วยความจำ{\wbr}หลาย{\wbr}ชนิด{\wbr}อาจ{\wbr}จะ{\wbr}มี{\wbr}ขนาด{\wbr}ใหญ่{\wbr}เกิน{\wbr}กว่า{\wbr}จะ{\wbr}เก็บ{\wbr}ได้{\wbr}ใน 1 ไบท์
เช่น{\wbr}ข้อมูล{\wbr}ประเภท {\ttt int} ใน{\wbr}ภาษา C/C++
โดยมาก{\wbr}ข้อมูล{\wbr}เหล่านั้น{\wbr}ก็{\wbr}จะ{\wbr}ถูก{\wbr}จัด{\wbr}เก็บ{\wbr}อยู่{\wbr}ใน{\wbr}หลาย{\wbr}ไบท์{\wbr}เรียง{\wbr}ต่อ{\wbr}กัน{\wbr}
รูปแบบ{\wbr}และ{\wbr}วิธีการ{\wbr}จัด{\wbr}เก็บ{\wbr}เหล่านี้{\wbr}อยู่{\wbr}นอก{\wbr}ขอบเขต{\wbr}ของ{\wbr}หนังสือ{\wbr}เล่ม{\wbr}นี้{\wbr}
ผู้อ่าน{\wbr}ที่{\wbr}สนใจ{\wbr}สามารถ{\wbr}อ่าน{\wbr}เพิ่มเติม{\wbr}ได้{\wbr}ใน{\wbr}หนังสือ{\wbr}สถาปัตยกรรม{\wbr}คอมพิวเตอร์{\wbr}ทั่วไป{\wbr}

ข้อมูล{\wbr}ประเภท{\wbr}พอยน์เตอร์{\wbr}เป็น{\wbr}ข้อมูล{\wbr}สำหรับ{\wbr}เก็บ{\wbr}ตำแหน่ง{\wbr}ของ{\wbr}หน่วยความจำ{\wbr}
เพื่อ{\wbr}ใช้{\wbr}อ้าง{\wbr}ถึง{\wbr}ข้อมูล{\wbr}ที่อยู่{\wbr}ใน{\wbr}ตำแหน่ง{\wbr}ดังกล่าว ใน{\wbr}บาง{\wbr}ภาษา{\wbr}เช่น Java หรือ C\#
ก็{\wbr}มี{\wbr}การ{\wbr}ใช้{\wbr}งาน{\wbr}ข้อมูล{\wbr}ชนิด{\wbr}นี้{\wbr}ใน{\wbr}การ{\wbr}อ้าง{\wbr}ถึง{\wbr}ข้อมูล{\wbr}ที่{\wbr}เป็น{\wbr}วัตถุ{\wbr}แต่{\wbr}เรียก{\wbr}ว่า{\wbr}เป็น{\wbr}การ{\wbr}อ้าง{\wbr}ถึง (reference)

วิธีการ{\wbr}ประกาศ{\wbr}และ{\wbr}ใช้{\wbr}ข้อมูล{\wbr}ประเภท{\wbr}นี้{\wbr}ขึ้น{\wbr}กับ{\wbr}ภาษา{\wbr}โปรแกรม{\wbr}ที่{\wbr}ใช้{\wbr}
สำหรับ{\wbr}การ{\wbr}ทำ{\wbr}ความ{\wbr}เข้าใจ{\wbr}ทั่วไป ข้อมูล{\wbr}ประเภท{\wbr}นี้{\wbr}มัก{\wbr}เขียน{\wbr}แทน{\wbr}ด้วย{\wbr}ช่อง{\wbr}ที่{\wbr}มี{\wbr}ลูกศร{\wbr}
เพื่อ{\wbr}แสดง{\wbr}การ{\wbr}ชี้{\wbr}ไป{\wbr}ยัง{\wbr}ตำแหน่ง{\wbr}ของ{\wbr}ข้อมูล{\wbr}อื่น ๆ
ดัง{\wbr}ตัวอย่าง{\wbr}ใน{\wbr}รูป~\ref{fig:array-example-pointer}

\begin{figure}
TODO: เพิ่ม{\wbr}รูป{\wbr}นี้{\wbr}
\caption{ตัวอย่าง{\wbr}การ{\wbr}เขียน{\wbr}พอยน์เตอร์}
\label{fig:array-example-pointer}
\end{figure}

\section{อาร์เรย์{\wbr}และ{\wbr}พอยน์เตอร์{\wbr}ใน{\wbr}ภาษา C/C++}
\label{sect:array-array-pointer-c}

\section{แบบฝึกหัด}

\begin{enumerate}
\item เขียน{\wbr}โปรแกรม{\wbr}ลำ{\wbr}ลอง{\wbr}สำหรับ{\wbr}การ{\wbr}แทรก{\wbr}ข้อมูล{\wbr}ใน{\wbr}รายการ{\wbr}
\end{enumerate}
